\documentclass{ximera}

\input{../preamble.tex}

\title[Dig-In:]{Quadric surfaces}

\begin{document}
\begin{abstract}
  
\end{abstract}
\maketitle

As we have seen, if we look at the set of points that satisfy an
equation
\[
F(x,y,z)=0
\]
where $F:\R^3\to\R$, we obtain a surface in $\R^3$. A basic class of
surfaces to know are the \textit{quadric surfaces}.

\begin{definition}
A \dfn{quadric surface} in $\R^3$ is a surface of the form
\[
\text{I'm NOT SURE WHAT THE BEST THING TO DO HERE IS}
\]
\end{definition}

\section{Elliptic paraboloid}

\section{Hyperbolic paraboloid}


\begin{warning}
  Do not confuse a \textit{quadric} with a quadratic, or quartic, as
  these are different beasts entirely.
\end{warning}

\begin{sageCell}
var ('x y z')
A = 3
B = 4
C = 1
implicit_plot3d(A*x^2+B*y^2-C*z^2==1,
                (x,-6,6), 
                (y,-6,6), 
                (z,-6, 6), 
                axes="true")
\end{sageCell}

\end{document}
