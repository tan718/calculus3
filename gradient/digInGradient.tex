\documentclass{ximera}

\input{../preamble.tex}


\title[Dig-In:]{The gradient}

\begin{document}
\begin{abstract}
We inroduce the gradient vector. 
\end{abstract}
\maketitle

The gradient is a useful tool in vector calculus. In some sense, it
has the role that the derivative has in single-variable calcululs.
Let's state the definition:

\begin{definition}
  Let $F:\R^n\to \R$ be a differentiable function, the \dfn{gradient}
  \[
  \grad F = \vector{\pp[F]{x_1},\pp[F]{x_2},\dots,\pp[F]{x_n}}
  \]
  is a vector-valued function of $n$ variables. 
\end{definition}

Let's see if you get this:


\begin{question}
  Let $F(x,y) = \sin(x)\cos(y)$, compute:
  \[
  \grad F(x,y)
  \begin{prompt}
    = \vector{\answer{\cos(x)\cos(y)},\answer{-\sin(x)\sin(y)}}
  \end{prompt}
  \]
  \begin{question}
    Let $\vec{p}= \vector{\pi/3,\pi/3}$. Compute:
    \[
    \grad F(\vec{p})
    \begin{prompt}
      =\vector{\answer{1/4},\answer{-3/4}}
    \end{prompt}
    \]
  \end{question}
\end{question}

And now in three variables:

\begin{question}
  Let $F(x,y) = ze^{-7xy}$, compute:
  \[
  \grad F(x,y)
  \begin{prompt}
    = \vector{\answer{z e^{-7xy} -7y},\answer{z e^{-7xy} -7x}, \answer{e^{-7xy}}}
  \end{prompt}
  \]
  \begin{question}
    Let $\vec{p}= \vector{1,0,1/7}$. Compute:
    \[
    \grad F(\vec{p})
    \begin{prompt}
      =\vector{\answer{0},\answer{-1},\answer{1}}
    \end{prompt}
    \]
  \end{question}
\end{question}

Now that we can compute the gradient, let's see if we can figure out
what it means.

\section{The greatest initial increase}

First recall what it means for a function to be \textit{differentiable}:
\begin{quote}\index{tangent plane}%%BADBAD would like an image
  Given a function $F:\R^\to\R$ and a vector $\vec{a}$ in the domain
  of $F$, if one can ``zoom in'' on the graph at $(\vec{a}, F(\vec{a}))$
  sufficiently so that it appears to be a plane, then the
  function is \dfn{differentiable}, and that plane is the \dfn{tangent plane}
  to $F$ at the point $(\vec{a},F(\vec{a}))$.
\end{quote}

Now let's imagine what the gradient is telling us about the
plane. First let $\vec{p}=\vector{a,b}$ and set
\[
\vector{m_1,m_2} = \grad F(\vec{p})
\]
Now the tangent plane is given by: 
\[
z = m_1 (x-a) + m_2 (y-b) + F(a,b)
\]
what happens if we leave the point $\vec{p}$ in the direction of
\[
\vector{m_1,m_2} = \grad F(\vec{p})?
\]
Well,
\begin{itemize}
  \item if a componet of $\grad F(\vec{p})$ is positive, travling in
    its direction will ensure that you are ``traveling uphill,''
    meaning you are raising the $z$-value (at least initially).
\item if a componet of $\grad F(\vec{p})$ is negative, travling in its
  (negative) direction will also ensure that you are ``traveling
  uphill,'' meaning you are raising the $z$-value (at least
  initially).
\end{itemize}
The componets of $\grad F(\vec{p})$ work together to ensure you are
traveling in the direction of the greatest initial increase.
\begin{onlineOnly}
For your viewing pleasure, we have included a graph where you can see
the $x$-component of the gradient and the $y$-component of the
gradient combining to show the direction one should leave a point, and
find the greatest initial increase in the function.
\begin{sageCell}
f(x,y) = 3*x+4*y
vectorx=arrow3d((0,0,0),(3,0,0),7,color='blue');
vectory=arrow3d((0,0,0),(0,4,0),7,color='blue');
grad=arrow3d((0,0,0),(3,4,0),7,color='red');

plot3d( f, (x,-3,3), (y,-3,3) ) + vectorx + vectory + grad
\end{sageCell}
\end{onlineOnly}


\section{Perpendicularity and the gradient}

\section{Formulas and the gradient}

\end{document}
