\documentclass{ximera}

\input{../preamble.tex}

\title[Dig-In:]{Curl and line integrals}

\begin{document}
\begin{abstract}
Green's Theorem is a fundamental theorem of calculus.
\end{abstract}
\maketitle

In this section we will learn the \textit{fundamental derivative} for
vector fields, as well as a new fundamental theorem of calculus.


\section{The curl of a vector field}

Calculus has taught us that knowing the derivative of a function
$f:\R\to\R$ can tell us important information about the function.  In
a similar way we have that seen that if we wish to understand a
function of several variables $F:\R^n\to\R$, then the gradient, $\grad
F$, contains similar useful information. But what if you have a vector
field
\[
\vec{F}:\R^n\to\R^n
\]
what is the natural analogue of a derivative in this setting? We will
give the answer when the vector field is two or three dimensional. You
can take another course to learn more about deritivatives of
$n$-dimensional vector fields.


\begin{definition}
  In two-dimensions, given a vector field $\vec{F}:\R^2\to \R^2$, where
  \[
  \vec{F}(x,y) = \vector{M(x,y),N(x,y)}
  \]
  the \dfn{curl} is given by
  \[
  \curl \vec F = \pp[N]{x}-\pp[M]{y}.
  \]
  In three-dimensions, given a vector field $\vec{F}:\R^3\to\R^3$< where
  \[
  \vec{F}(x,y,z) = \vector{U(x,y,z)),V(x,y,z)),W(x,y,z)}
  \]
  the \dfn{curl} is given by
  \begin{align*}
  \curl \vec F &= \det
  \begin{bmatrix}
    \veci & \vecj & \veck \\
    \pp{x} & \pp{y} & \pp{z}\\
    U & V & W
  \end{bmatrix}\\
  &= \veci\left(\pp[W]{y}-\pp[V]{z}\right)-
  \vecj\left(\pp[W]{x}-\pp[U]{z}\right)+
  \veck\left(\pp[V]{x}-\pp[U]{y}\right).
  \end{align*}
\end{definition}

\begin{question}
  In two dimensions $\curl\vec{F}$ is a
  \begin{multipleChoice}
    \choice[correct]{number.}
    \choice{vector.}
  \end{multipleChoice}
  \begin{question}
    In three dimensions $\curl\vec{F}$ is a
    \begin{multipleChoice}
      \choice{number.}
      \choice[correct]{vector.}
    \end{multipleChoice}
  \end{question}
\end{question}


\begin{question}
  Consider the vector field $\vec{F}(x,y) = \vector{-y,x}$. Compute:
  \[
  \curl\vec{F}(x,y) \begin{prompt}= \answer{2}\end{prompt}
  \]
  \begin{question}
    Consider the vector field $\vec{F}(x,y,z) = \vector{-z,x,y}$. Compute:
    \[
    \curl\vec{F}(x,y,z)   \begin{prompt}
      = \vector{\answer{1},\answer{1},\answer{1}}
    \end{prompt}
    \]
  \end{question}
\end{question}

Now for something you've seen before, but in a different form.

\begin{question}
  Let $F:\R^2\to\R$. Compute:
  \[
  \curl\grad F(x,y) \begin{prompt}= \answer{0}\end{prompt}
  \]
  \begin{question}
    Let $F:\R^3\to\R$. Compute:
    \[
    \curl\grad F(x,y,z) \begin{prompt}= \vector{\answer{0},\answer{0},\answer{0}}\end{prompt}
    \] 
  \end{question}
\end{question}

\begin{question}
  When $\curl\vec{F} = \vec{0}$, then you know:
  \begin{selectAll}
    \choice[correct]{$\vec{F}$ is a gradient field.}
    \choice[correct]{$\vec{F}$ is a conserttive field.}
    \choice[correct]{$\vec{F}:\R^3\to\R^3$.}
  \end{selectAll}
  \begin{feedback}
    You can be assured that $\vec{F}:\R^3\to\R^3$, since $\vec{0}$ is
    a vector. We only know a definition for curl in two and three
    dimensions; however, the two dimensional definition is a scalar,
    not a vector. So if $\curl\vec{F} = \vec{0}$, then $\vec{F}:\R^3\to\R^3$.
  \end{feedback}
\end{question}


\subsection{What does the curl measure?}


The curl of a vector field measures the rate that the direction of
field vectors ``twist'' as $x$ and $y$ change. Unfortunately, while we
can sometimes identify curl from a graph.

beachball?


\begin{question}
  Discrete example use field $\vector{0,x}$
\end{question}



\[
\{F:\R^2 \to \R\} \lto^{\grad} \{\vec{F}: \R^2 \to \R^2\} \lto^{\curl}
\{F:\R^2 \to \R\} 
\]



Recall,

\section{A new fundamental theorem of calculus}

\begin{theorem}[Green's Theorem]
  If $\vec{F}$ has continuous partial derivatives and $C$ is a
  boundray of a closed region $R$ and $\vec{p}(t)$ paramaterizes $C$
  in a counterclockwise direction with the interior on the left, then
  \[
  \int_R \curl\vec{F}\d A = \int_C \vec{F}\dotp\d\vec{p} 
  \]
\end{theorem}

\end{document}
