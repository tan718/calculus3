\documentclass{ximera}

%\usepackage{todonotes}

\newcommand{\todo}{}


\graphicspath{
{./}
{../functionsOfSeveralVariables/}
{../normalVectors/}
{../lagrangeMultipliers/}
{../vectorFields/}
{../greensTheorem/}
}


\usepackage{tkz-euclide}
\tikzset{>=stealth} %% cool arrow head
\tikzset{shorten <>/.style={ shorten >=#1, shorten <=#1 } } %% allows shorter vectors

\usetikzlibrary{backgrounds} %% for boxes around graphs
\usetikzlibrary{shapes,positioning}  %% Clouds and stars
\usetikzlibrary{matrix} %% for matrix
\usepgfplotslibrary{polar} %% for polar plots
\usetkzobj{all}
\usepackage[makeroom]{cancel} %% for strike outs
%\usepackage{mathtools} %% for pretty underbrace % Breaks Ximera
\usepackage{multicol}
\usepackage{pgffor} %% required for integral for loops


%% http://tex.stackexchange.com/questions/66490/drawing-a-tikz-arc-specifying-the-center
%% Draws beach ball
\tikzset{pics/carc/.style args={#1:#2:#3}{code={\draw[pic actions] (#1:#3) arc(#1:#2:#3);}}}



\usepackage{array}
\setlength{\extrarowheight}{+.1cm}   
\newdimen\digitwidth
\settowidth\digitwidth{9}
\def\divrule#1#2{
\noalign{\moveright#1\digitwidth
\vbox{\hrule width#2\digitwidth}}}





\newcommand{\RR}{\mathbb R}
\newcommand{\R}{\mathbb R}
\newcommand{\N}{\mathbb N}
\newcommand{\Z}{\mathbb Z}

\newcommand{\sage}{\textsf{SageMath}}


%\renewcommand{\d}{\,d\!}
\renewcommand{\d}{\mathop{}\!d}
\newcommand{\dd}[2][]{\frac{\d #1}{\d #2}}
\newcommand{\pp}[2][]{\frac{\partial #1}{\partial #2}}
\renewcommand{\l}{\ell}
\newcommand{\ddx}{\frac{d}{\d x}}

\newcommand{\zeroOverZero}{\ensuremath{\boldsymbol{\tfrac{0}{0}}}}
\newcommand{\inftyOverInfty}{\ensuremath{\boldsymbol{\tfrac{\infty}{\infty}}}}
\newcommand{\zeroOverInfty}{\ensuremath{\boldsymbol{\tfrac{0}{\infty}}}}
\newcommand{\zeroTimesInfty}{\ensuremath{\small\boldsymbol{0\cdot \infty}}}
\newcommand{\inftyMinusInfty}{\ensuremath{\small\boldsymbol{\infty - \infty}}}
\newcommand{\oneToInfty}{\ensuremath{\boldsymbol{1^\infty}}}
\newcommand{\zeroToZero}{\ensuremath{\boldsymbol{0^0}}}
\newcommand{\inftyToZero}{\ensuremath{\boldsymbol{\infty^0}}}



\newcommand{\numOverZero}{\ensuremath{\boldsymbol{\tfrac{\#}{0}}}}
\newcommand{\dfn}{\textbf}
%\newcommand{\unit}{\,\mathrm}
\newcommand{\unit}{\mathop{}\!\mathrm}
\newcommand{\eval}[1]{\bigg[ #1 \bigg]}
\newcommand{\seq}[1]{\left( #1 \right)}
\renewcommand{\epsilon}{\varepsilon}
\renewcommand{\iff}{\Leftrightarrow}

\DeclareMathOperator{\arccot}{arccot}
\DeclareMathOperator{\arcsec}{arcsec}
\DeclareMathOperator{\arccsc}{arccsc}
\DeclareMathOperator{\si}{Si}
\DeclareMathOperator{\proj}{\vec{proj}}
\DeclareMathOperator{\scal}{scal}
\DeclareMathOperator{\sign}{sign}


%% \newcommand{\tightoverset}[2]{% for arrow vec
%%   \mathop{#2}\limits^{\vbox to -.5ex{\kern-0.75ex\hbox{$#1$}\vss}}}
\newcommand{\arrowvec}{\overrightarrow}
%\renewcommand{\vec}[1]{\arrowvec{\mathbf{#1}}}
\renewcommand{\vec}{\mathbf}
\newcommand{\veci}{{\boldsymbol{\hat{\imath}}}}
\newcommand{\vecj}{{\boldsymbol{\hat{\jmath}}}}
\newcommand{\veck}{{\boldsymbol{\hat{k}}}}
\newcommand{\vecl}{\boldsymbol{\l}}
\newcommand{\utan}{\mathbf{\hat{t}}}
\newcommand{\unormal}{\mathbf{\hat{n}}}
\newcommand{\ubinormal}{\mathbf{\hat{b}}}

\newcommand{\dotp}{\bullet}
\newcommand{\cross}{\boldsymbol\times}
\newcommand{\grad}{\boldsymbol\nabla}
\newcommand{\divergence}{\grad\dotp}
\newcommand{\curl}{\grad\cross}
%\DeclareMathOperator{\divergence}{divergence}
%\DeclareMathOperator{\curl}[1]{\grad\cross #1}
\newcommand{\lto}{\mathop{\longrightarrow\,}\limits}


\colorlet{textColor}{black} 
\colorlet{background}{white}
\colorlet{penColor}{blue!50!black} % Color of a curve in a plot
\colorlet{penColor2}{red!50!black}% Color of a curve in a plot
\colorlet{penColor3}{red!50!blue} % Color of a curve in a plot
\colorlet{penColor4}{green!50!black} % Color of a curve in a plot
\colorlet{penColor5}{orange!80!black} % Color of a curve in a plot
\colorlet{fill1}{penColor!20} % Color of fill in a plot
\colorlet{fill2}{penColor2!20} % Color of fill in a plot
\colorlet{fillp}{fill1} % Color of positive area
\colorlet{filln}{penColor2!20} % Color of negative area
\colorlet{fill3}{penColor3!20} % Fill
\colorlet{fill4}{penColor4!20} % Fill
\colorlet{fill5}{penColor5!20} % Fill
\colorlet{gridColor}{gray!50} % Color of grid in a plot

\newcommand{\surfaceColor}{violet}
\newcommand{\surfaceColorTwo}{redyellow}
\newcommand{\sliceColor}{greenyellow}




\pgfmathdeclarefunction{gauss}{2}{% gives gaussian
  \pgfmathparse{1/(#2*sqrt(2*pi))*exp(-((x-#1)^2)/(2*#2^2))}%
}


%%%%%%%%%%%%%
%% Vectors
%%%%%%%%%%%%%

%% Simple horiz vectors
\renewcommand{\vector}[1]{\left\langle #1\right\rangle}


%% %% Complex Horiz Vectors with angle brackets
%% \makeatletter
%% \renewcommand{\vector}[2][ , ]{\left\langle%
%%   \def\nextitem{\def\nextitem{#1}}%
%%   \@for \el:=#2\do{\nextitem\el}\right\rangle%
%% }
%% \makeatother

%% %% Vertical Vectors
%% \def\vector#1{\begin{bmatrix}\vecListA#1,,\end{bmatrix}}
%% \def\vecListA#1,{\if,#1,\else #1\cr \expandafter \vecListA \fi}

%%%%%%%%%%%%%
%% End of vectors
%%%%%%%%%%%%%

%\newcommand{\fullwidth}{}
%\newcommand{\normalwidth}{}



%% makes a snazzy t-chart for evaluating functions
%\newenvironment{tchart}{\rowcolors{2}{}{background!90!textColor}\array}{\endarray}

%%This is to help with formatting on future title pages.
\newenvironment{sectionOutcomes}{}{} 



%% Flowchart stuff
%\tikzstyle{startstop} = [rectangle, rounded corners, minimum width=3cm, minimum height=1cm,text centered, draw=black]
%\tikzstyle{question} = [rectangle, minimum width=3cm, minimum height=1cm, text centered, draw=black]
%\tikzstyle{decision} = [trapezium, trapezium left angle=70, trapezium right angle=110, minimum width=3cm, minimum height=1cm, text centered, draw=black]
%\tikzstyle{question} = [rectangle, rounded corners, minimum width=3cm, minimum height=1cm,text centered, draw=black]
%\tikzstyle{process} = [rectangle, minimum width=3cm, minimum height=1cm, text centered, draw=black]
%\tikzstyle{decision} = [trapezium, trapezium left angle=70, trapezium right angle=110, minimum width=3cm, minimum height=1cm, text centered, draw=black]


\title[Dig-In:]{Curl and line integrals}

\begin{document}
\begin{abstract}
Green's Theorem is a fundamental theorem of calculus.
\end{abstract}
\maketitle

In this section we will learn the \textit{fundamental derivative} for
vector fields, as well as a new fundamental theorem of calculus.


\section{The curl of a vector field}

Calculus has taught us that knowing the derivative of a function
$f:\R\to\R$ can tell us important information about the function.  In
a similar way we have that seen that if we wish to understand a
function of several variables $F:\R^n\to\R$, then the gradient, $\grad
F$, contains similar useful information. But what if you have a vector
field
\[
\vec{F}:\R^n\to\R^n,
\]
what is the natural analogue of a derivative in this setting? When the
vector field is two or three dimensional, the \textit{curl} is the
analogue of the derivative that we are looking for:


\begin{definition}
  In two-dimensions, given a vector field $\vec{F}:\R^2\to \R^2$, where
  \[
  \vec{F}(x,y) = \vector{M(x,y),N(x,y)}
  \]
  the \dfn{curl} is given by
  \[
  \curl \vec F = \pp[N]{x}-\pp[M]{y}.
  \]
  In three-dimensions, given a vector field $\vec{F}:\R^3\to\R^3$, where
  \[
  \vec{F}(x,y,z) = \vector{U(x,y,z)),V(x,y,z)),W(x,y,z)}
  \]
  the \dfn{curl} is given by
  \begin{align*}
  \curl \vec F &= \det
  \begin{bmatrix}
    \veci & \vecj & \veck \\
    \pp{x} & \pp{y} & \pp{z}\\
    U & V & W
  \end{bmatrix}\\
  &= \veci\left(\pp[W]{y}-\pp[V]{z}\right)-
  \vecj\left(\pp[W]{x}-\pp[U]{z}\right)+
  \veck\left(\pp[V]{x}-\pp[U]{y}\right).
  \end{align*}
\end{definition}


\begin{question}
  In two dimensions $\curl\vec{F}$ is a
  \begin{multipleChoice}
    \choice[correct]{number.}
    \choice{vector.}
  \end{multipleChoice}
  \begin{question}
    In three dimensions $\curl\vec{F}$ is a
    \begin{multipleChoice}
      \choice{number.}
      \choice[correct]{vector.}
    \end{multipleChoice}
  \end{question}
\end{question}


\begin{question}
  Consider the vector field $\vec{F}(x,y) = \vector{-y,x}$. Compute:
  \[
  \curl\vec{F}(x,y) \begin{prompt}= \answer{2}\end{prompt}
  \]
  \begin{question}
    Consider the vector field $\vec{F}(x,y,z) = \vector{-z,x,y}$. Compute:
    \[
    \curl\vec{F}(x,y,z)   \begin{prompt}
      = \vector{\answer{1},\answer{1},\answer{1}}
    \end{prompt}
    \]
  \end{question}
\end{question}

Now for something you've seen before, but in a different form.

\begin{question}
  Let $F:\R^2\to\R$. Compute:
  \[
  \curl\grad F(x,y) \begin{prompt}= \answer{0}\end{prompt}
  \]
  \begin{question}
    Let $F:\R^3\to\R$. Compute:
    \[
    \curl\grad F(x,y,z) \begin{prompt}= \vector{\answer{0},\answer{0},\answer{0}}\end{prompt}
    \] 
  \end{question}
\end{question}

\begin{question}
  When $\curl\vec{F} = \vec{0}$, then you know:
  \begin{selectAll}
    \choice[correct]{$\vec{F}$ is a gradient field.}
    \choice[correct]{$\vec{F}$ is a conservative field.}
    \choice[correct]{$\vec{F}:\R^3\to\R^3$.}
  \end{selectAll}
  \begin{feedback}
    You can be assured that $\vec{F}:\R^3\to\R^3$, since $\vec{0}$ is
    a vector. We only know a definition for curl in two and three
    dimensions; however, the two dimensional definition is a scalar,
    not a vector. So if $\curl\vec{F} = \vec{0}$, then $\vec{F}:\R^3\to\R^3$.
  \end{feedback}
\end{question}


\subsection{What does the curl measure?}


The curl of a vector field measures the rate that the direction of
field vectors ``twist'' as $x$ and $y$ change. Imagine the vectors in
a vector field as representing the current of a river. A positive curl
at a point tells you that a ``beach-ball'' floating at the point would
be rotating in a counterclockwise direction. A negative curl at a
point tells you that a ``beach-ball'' floating at the point would be
rotating in a clockwise direction. Zero curl means that the
``beach-ball'' would not be rotating. Below we see our ``beach-ball''
with two field vectors. If $\vec{F}(x,y) = \vector{M(x,y),N(x,y)}$
\begin{image}
  \begin{tikzpicture}%[framed]
    \begin{axis}[
        hide axis,
        width=2in,
        height=2in,
	ymin=-.9,ymax=.7,
	xmin=-.8,xmax=.8,
      ]
       \addplot[penColor2,thick,->] coordinates{
                (-.6,0) (-.6,.3) 
       };
       \addplot[penColor2,thick,->] coordinates{
                (.6,0) (.6,.7) 
       };


      \addplot[penColor, ultra thick, domain=0:360,smooth] ({.5*cos(x)},{.5*sin(x)});
      \draw[penColor, ultra thick,->] (axis cs: .49,-.1) -- (axis cs: .5,.1) ;
      \node[inner sep=0pt,text width=6cm,scale=.5] at (axis cs: 0,-.7) {

        \footnotesize Since the length of the field vector increases
        as we increase $x$, we see that $\pp[N]{x} > 0$.};

    \end{axis}
  \end{tikzpicture}
\end{image}
we see that the right field vector is larger than the left, thus
giving the ``beach-ball'' a counterclockwise rotation.  In an entirely
similar way, if we have
\begin{image}
  \begin{tikzpicture}%[framed]
    \begin{axis}[
        hide axis,
        width=2in,
        height=2in,
	ymin=-.9,ymax=.7,
	xmin=-.8,xmax=.8,
      ]
       \addplot[penColor2,thick,->] coordinates{
                (0,.6) (.3,.6) 
       };
       \addplot[penColor2,thick,->] coordinates{
                (0,-.6) (.7,-.6) 
            };


      \addplot[penColor, ultra thick, domain=0:360,smooth] ({.5*cos(x)},{.5*sin(x)});
      \draw[penColor, ultra thick,->] (axis cs: .49,-.1) -- (axis cs: .5,.1) ;
      \node[inner sep=0pt,text width=6cm,scale=.5] at (axis cs: 0,-.75) {

        \footnotesize Since the length of the field vector decreases
        as we increase $y$, we see that $-\pp[M]{y} > 0$.};
    \end{axis}
  \end{tikzpicture}
\end{image}
we see that the bottom field vector is larger than the top, thus
giving the ``beach-ball'' a counterclockwise rotation. Thus the curl
combines $\pp[N]{x}$ and $-\pp[M]{y}$
\[
\curl\vec{F} = \pp[N]{x} - \pp[M]{y}
\]
to obtain the infinitesimal rotation of the field. The most obvious
example of a vector field with nonzero curl is $\vec{F}(x,y) =
\vector{-y,x}$.
\begin{image}
  \includegraphics{rotField.png}
\end{image}
Unfortunately, while we can sometimes identify nonzero curl from a
graph, it can be difficult. In our next example, we see a field that
does not have global rotation but does have local rotation.

\begin{example}
  Consider the following vector field $\vec{F}$:
      \begin{image}
        \begin{tikzpicture}
          \begin{axis}%
            [
	      ymin=-2.2,ymax=2.2,
	      xmin=-3.2,xmax=3.2,
              axis lines =middle, xlabel=$x$, ylabel=$y$,
              every axis y label/.style={at=(current axis.above origin),anchor=south},
              every axis x label/.style={at=(current axis.right of origin),anchor=west},
              grid=both,
              grid style={dashed, gridColor},
              xtick={-3,...,3},
              ytick={-2,...,2},
	    ]
            \pgfplotsinvokeforeach{-2,...,2}{
              \fill[penColor] (axis cs:0,#1) circle (2pt);
              }
            \pgfplotsinvokeforeach{-2,...,1}{
              \addplot[penColor,thick,->] coordinates{
                (1,#1+.05) (1,#1+.95) 
              };
            }
            \pgfplotsinvokeforeach{-2,...,1}{
              \addplot[penColor,thick,->] coordinates{
                (-1,.95 + #1) (-1,#1+.05) 
              };
            }
            \pgfplotsinvokeforeach{-2,0}{
              \addplot[penColor,thick,->] coordinates{
                (2,#1+.05) (2,#1+1.95) 
              };
            }
            \pgfplotsinvokeforeach{-2,0}{
              \addplot[penColor,thick,->] coordinates{
                (-2,1.95 + #1) (-2,#1+.05) 
              };
            }
            \addplot[penColor,thick,->] coordinates{
                (3,-1.5) (3,1.5) 
              };
            \addplot[penColor,thick,->] coordinates{
                (-3,1.5) (-3,-1.5) 
            };

            \fill[black,draw=black] (axis cs:2,1) circle (2.5pt);

            \node[right] at (axis cs:2,1) {$\vec{a}$};
              %% \node[inner sep=0pt,text width=8cm,right,scale=.85] at (axis cs:-6,-3.5)
              %%      {\footnotesize One should imagine a vector at
              %%        \textbf{every} point. We'll assume that the magnitudes
              %%        of the vectors are constant along horizontal lines.};
          \end{axis}
        \end{tikzpicture}
      \end{image}
      Setting $\d x = 1$ and $\d y = 1$, estimate:
      \[
      \curl \vec{F}(\vec{a})
      \]
      \begin{explanation}
        First note that one should imagine a vector at \textbf{every}
        point. We'll assume that the magnitudes of the vectors are
        constant along vertical lines. Set $\vec{F}(x,y) =
        \vector{M(x,y),N(x,y)}$. To estimate $\pp[N]{x}$, we examine
        the change in $N(x,1)$ between $x=0$ and $x=1$:
        \[
        \frac{N(2,1) - N(1,1)}{2-1} = \answer[given]{1}
        \]
        and we should also check the change of $N(x,1)$ between $x=2$
        and $x=3$:
        \[
        \frac{N(3,1) - N(2,1)}{3-2} = \answer[given]{1}
        \]
        Averaging these values together we find
        \[
        \pp[N]{x} \approx \answer[given]{1}
        \]
        To estimate $\pp[M]{y}$, we examine the change in $M(2,y)$
        between $y=0$ and $y=1$:
        \[
        \frac{M(2,1) - M(2,0)}{1-0} = \answer[given]{0}
        \]
        and we should also check the change of $M(2,y)$ between $y=1$
        and $y=2$:
        \[
        \frac{M(2,2) - M(2,1)}{2-1} = \answer[given]{0}
        \]
        Averaging these values together we find
        \[
        \pp[M]{y} \approx \answer[given]{0}
        \]
        So we approximate
        \[
        \curl\vec{F}(\vec{a})\approx\answer[given]{1}.
        \]
        So field above example does not have global rotation, but it
        does have local rotation.
      \end{explanation}
\end{example}

Now we'll show you a field that has global rotation but no local rotation!

\begin{example}
  Consider the vector field
  \[
  \vec{F}(x,y) = \vector{\frac{-y}{x^2+y^2},\frac{x}{x^2+y^2}}
  \]
  \begin{image}
    \includegraphics{rotFieldNot.png}
  \end{image}
  Compute:
  \[
  \curl\vec{F}(x,y)
  \]
  \begin{explanation}
    In this case, $\curl\vec{F}(x,y) =\answer[given]{0}$ when
    $\vector{x,y}\ne \vec{0}$. At $\vec{0}$, the curl of the vector
    field is
    \wordChoice{\choice{zero}\choice{infinity}\choice[correct]{undefined}}.
    This field has global rotation, but it does not have local
    rotation.
  \end{explanation}
\end{example}

Finally, we'll show you a field that has global rotation with local
rotation in the \textit{opposite} direction!
\begin{example}
  Consider the vector field
  \[
  \vec{F}(x,y) = \vector{\frac{-y}{(x^2+y^2)^{3/2}},\frac{x}{(x^2+y^2)^{3/2}}}
  \]
  \begin{image}
    \includegraphics{oppoRotField.png}
  \end{image}
  Compute:
  \[
  \curl\vec{F}(x,y)
  \]
  \begin{explanation}
    In this case,
    \[
    \curl\vec{F}(x,y) =\answer[given]{\frac{-1}{(x^2+y^2)^{3/2}}}
    \]
    when $\vector{x,y}\ne \vec{0}$. Note, this is
    \wordChoice{\choice{positive}\choice[correct]{negative}} for all
    values of $x$ and $y$. This corresponds to local rotation in the
    \wordChoice{\choice[correct]{clockwise}\choice{counterclockwise}}
    direction.  This field has global rotation in the
    \wordChoice{\choice{clockwise}\choice[correct]{counterclockwise}}
    direction, but has local rotation in the
    \wordChoice{\choice[correct]{clockwise}\choice{counterclockwise}}
    direction.
  \end{explanation}
\end{example}

At this point we only know how to take the derivative (via the curl)
of a vector field of two or three dimensions.  You can take another
course to learn more about derivatives of $n$-dimensional vector
fields.

\section{A new fundamental theorem of calculus}

Recall that a fundamental theorem of calculus says something like:
\begin{quote}
  To compute a certain sort of integral over a region, we may do a
  computation on the boundary of the region that involves one fewer
  integrations.
\end{quote}
In the single variable case we have:
\begin{image}
  \begin{tikzpicture}
    \node[inner sep=0pt] at (0,0) {$\int_a^b f'(x) \d x = f(b) - f(a)$};

    \node at (-1.1,-.7) {$\underbrace{\hspace{5.8em}}$};

    \node at (1.1,-.7) {$\underbrace{\hspace{5.5em}}$};

    \node[below,inner sep=0pt,text width=4cm,scale=.5] at (-1.25,-1) {To compute the signed area between $f'$ and the $x$-axis on the interval $[a,b]$,};

    \node[below,inner sep=0pt,text width=4cm,scale=.5] at (1.25,-1) {we can compute the difference in the values of $f$ when evaluated on the boundry.};
  \end{tikzpicture}
\end{image}
With line integrals we have:
\begin{image}
  \begin{tikzpicture}
    \node[inner sep=0pt] at (0,0) {$\int_C \grad F \dotp  \d \vec{p} = F(\vec{b}) - F(\vec{a})$};

    \node at (-1.1,-.7) {$\underbrace{\hspace{5.5em}}$};

    \node at (1,-.7) {$\underbrace{\hspace{5.5em}}$};

    \node[below,inner sep=0pt,text width=4cm,scale=.5] at (-1.1,-1)
         {To compute the accumulation of $\grad F$ along a curve $C$
           starting at $\vec{a}$ and ending at $\vec{b}$,};

    \node[below,inner sep=0pt,text width=4cm,scale=.5] at (1.1,-1) {we can compute the difference in the values of $F$ when evaluated on the boundry.};
  \end{tikzpicture}
\end{image}
We now introduce a new fundamental theorem of calculus involving the
curl. It's called \textit{Green's Theorem}:

\begin{theorem}[Green's Theorem]\index{Green's Theorem}
  If the components of $\vec{F}:\R^2\to\R^2$ have continuous partial
  derivatives and $C$ is a boundary of a closed region $R$ and
  $\vec{p}(t)$ parameterizes $C$ in a counterclockwise direction with
  the interior on the left, then
  \[
  \iint_R \curl\vec{F}\d A = \int_C \vec{F}\dotp\d\vec{p} 
  \]
\end{theorem}

How is this a fundamental theorem of calculus? Well consider this:
\begin{image}
  \begin{tikzpicture}
    \node[inner sep=0pt] at (0,0) {$\iint_R \curl\vec{F}\d A = \int_C \vec{F}\dotp\d\vec{p}$};

    \node at (-.9,-.7) {$\underbrace{\hspace{5.7em}}$};

    \node at (1.1,-.7) {$\underbrace{\hspace{5em}}$};

    \node[below,inner sep=0pt,text width=4cm,scale=.5] at (-1.2,-1)
         {To compute the double intgeral of $\curl \vec{F}$ over a
           region $R\subseteq\R^2$ with boundry $C$, };

    \node[below,inner sep=0pt,text width=4cm,scale=.5] at (1.25,-1) {we can compute the accumulation of $\vec{F}$ along a boundry curve $C$.};
  \end{tikzpicture}
\end{image}




\end{document}
