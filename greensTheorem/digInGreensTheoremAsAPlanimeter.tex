\documentclass{ximera}

\input{../preamble.tex}

\title[Dig-In:]{Green's theorem as a planimeter}

\begin{document}
\begin{abstract}
  A planimeter computes the area of a region by tracing the boundry.
\end{abstract}
\maketitle

Green's theorem may seem rather abstract, but as we will see, it is a
fantastic tool for computing the areas of arbitrary bounded
regions. In particular, Green's Theorem is a theoritical
\link[planimeter]{http://en.wikipedia.org/wiki/Planimeter}.

A \dfn{planimeter} is a ``device'' used for measuring the area of a
region. Ideally, one would ``trace'' the border of a region, and the
planimeter would tell you the area of the region. 


\section{How is Green's theorem a planimeter?}

Recall Green's Theorem:
\begin{theorem}[Green's Theorem]\index{Green's Theorem}
  If the components of $\vec{F}:\R^2\to\R^2$ have continuous partial
  derivatives and $C$ is a boundary of a closed region $R$ and
  $\vec{p}(t)$ parameterizes $C$ in a counterclockwise direction with
  the interior on the left, then
  \[
  \iint_R \curl\vec{F}\d A = \int_C \vec{F}\dotp\d\vec{p} 
  \]
\end{theorem}

Given a vector field $\vec{F}:\R^2\to\R^2$, if $\curl\vec{F} = 1$,
then the left-hand side of the conclusion of Green's Theorem gives the
area of the region $R$:
\[
\iint_R \curl\vec{F}\d A = \iint_R \d A
\]

So now the question becomes, which vector fields have $\curl\vec{F} =
1$?

Here are three basic candidates:
\begin{itemize}
\item $\vec{F}(x,y) = \vector{0,x}$
\item $\vec{F}(x,y) = \vector{-y,0}$
\item $\vec{F}(x,y) = \vector{-y/2,x/2}$
\end{itemize}
\begin{question}
  The key idea that connects the three vector fields above is:
  \begin{selectAll}
    \choice[correct]{Their curl is $1$.}
    \choice{They are conservative fields.}
    \choice{They are gradient fields.}
    \choice[correct]{When used in combination with Green's Theorem, they help compute area.}
  \end{selectAll}
\end{question}

Once we have a vector field whose curl is $1$, we may then apply
Green's Theorem to use a line integral to compute the area.

\begin{warning}
  You must parameterize $C$ with $\vec{p}(t)$ for $a\le t\le b$ such that:
  \begin{itemize}
    \item $C$ is drawn in a counterclockwise direction.
    \item $C$ is drawn exactly once.
    \item The interior of $R$ is to the right of the direction of
      $\vec{p}'(t)$.
  \end{itemize}
\end{warning}




\section{Computing areas with Green's Theorem}

Now let's do some examples.

\begin{example}
  Compute the area of an $a\times b$ rectangle using Green's Theorem.
  \begin{explanation}
    In this case, set $F(x,y) = \vector{0,x}$. Since $\curl\vec{F} =
    1$, Green's Theorem says:
    \[
    \iint_R \d A  = \int_C \vector{0,x}\dotp\d\vec{p}
    \]
    Parameterizing the boundry of the region (an $a\times b$
    rectangle)
    \begin{image}
      rect
    \end{image}
    we see that 
    
  \end{explanation}
\end{example}


1) rectangle

2) ellipse

3) polygonal boundry



\end{document}
