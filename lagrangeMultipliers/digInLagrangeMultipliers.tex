\documentclass{ximera}

\input{../preamble.tex}


\title[Dig-In:]{Lagrange multipliers}

\begin{document}
\begin{abstract}
  We give a new method of finding extrema. 
\end{abstract}
\maketitle

Through out this course, we hope it has become apparent that when
given a problem:
\begin{quote}
  \textbf{There is more than one way to solve it.}
\end{quote}
Previously, we were finding extrema of functions $F:\R^n\to\R$ when
constrained to some set $S$.

\begin{example}
  Let $F(x,y) = x^2+3y^2-4x-6y+7$ and let $S$ the set
  \[
  S = \{(x,y):x^2 + y^2 \le 1\}
  \]
  Discuss how to find the maximum and minimum values of $F$ on $S$.
  \begin{explanation}
    To do this we first find the critical points of $F$, that is where
    the components of $\grad F(x,y)$ are zero or do not exist. If
    these points are in $S$, they become candidates for the maxima and
    minima.

    Next we investigate the boundary of $S$. To do this we must
    parametrize the boundary of $S$, or express the boundary either in
    terms of $x$ or $y$. In this case, we can express the boundary
    parametrically as
    \[
    \vec{p}(t) = \vector{\cos(t),\sin(t)},
    \]
    or we could set
    \[
    y = \pm \answer[given]{\sqrt{1-x^2}},
    \]
    or even
    \[
    x = \pm \answer[given]{\sqrt{1-y^2}}.
    \]
  \end{explanation}
\end{example}

Above we see that a crucial step in computing extrema when constrained
to a set is finding an \textbf{explicit formula} for the boundary. This
could be very difficult or even impossible! If our constraining set had been
\[
S = \{(x,y): x+y+\sin(xy) =0\}
\]
our previous method will not work, as we (at least this author!)
cannot find an explicit formula describing the boundary (which in this
case is the set) of the set above. Nevertheless, there is another
way. It is called the method of \textit{Lagrange multipliers}. This
method is named after the mathematician \link[Joesph-Louis
  Lagrange]{http://en.wikipedia.org/wiki/Joseph-Louis_Lagrange}, which
he first employed. This method relies on the \textit{gradient
  vector}. Recall facts about the gradient vector:
\begin{itemize}
\item $\grad F = \vector{\pp[F]{x_1},\pp[F]{x_2},\dots,\pp[F]{x_n}}$.
\item $\grad F(\vec{x})$ points in the direction that one must leave
  $\vec{x}$ in order to see the initial greatest increase in $F$.
\item $\grad F(\vec{x})$ points in the direction that is perpendicular
  to any level surface of $F$.
\end{itemize}

It is this last fact that we will look at now.  Below we see level
curves for some function $F:\R^2\to\R$ along with a constraining curve
that we will call $S$:
\begin{image}
  \begin{tikzpicture}
    \begin{axis}%
      [
        unit vector ratio*=1 1 1,
	ymin=-.2,ymax=4.5,
        width=5in,
	xmin=-4.5,xmax=4.5,
        axis lines=none,
      ]
      \addplot[ultra thick, penColor,smooth,domain=0:180] ({cos(x)},{sin(x)});
      \addplot[ultra thick, penColor,smooth,domain=0:180] ({2*cos(x)},{2*sin(x)});
      \addplot[ultra thick, penColor,smooth,domain=0:180] ({3*cos(x)},{3*sin(x)});
      \addplot[ultra thick, penColor,smooth,domain=0:180] ({4*cos(x)},{4*sin(x)});
      
      \addplot[ultra thick, penColor4,smooth] {x^2+1};      
      
      \node[penColor,fill=white] at (axis cs:-.7,.7) {$7$};
      \node[penColor,fill=white] at (axis cs:-1.4,1.4) {$6$};
      \node[penColor,fill=white] at (axis cs:-2.1,2.1) {$5$};
      \node[penColor,fill=white] at (axis cs:-2.8,2.8) {$4$};

      \node[penColor4] at (axis cs:2,4) {$S$};
      \node[penColor] at (axis cs:-3.8,2) {$F$};
    \end{axis}
  \end{tikzpicture}
\end{image}

Let's add vectors to our graph that point in the direction of $\grad
F(x,y)$. We can do this \textit{without} computation, since we
understand the geometry of the gradient vector, the gradient vector is
perpendicular to level curves:
\begin{image}
  \begin{tikzpicture}
    \begin{axis}%
      [
        unit vector ratio*=1 1 1,
	ymin=-.2,ymax=4.5,
        width=5in,
	xmin=-4.5,xmax=4.5,
        axis lines=none,
      ]
      \addplot[ultra thick, penColor,smooth,domain=0:180] ({cos(x)},{sin(x)});
      \addplot[ultra thick, penColor,smooth,domain=0:180] ({2*cos(x)},{2*sin(x)});
      \addplot[ultra thick, penColor,smooth,domain=0:180] ({3*cos(x)},{3*sin(x)});
      \addplot[ultra thick, penColor,smooth,domain=0:180] ({4*cos(x)},{4*sin(x)});
      
      \addplot[ultra thick, penColor4,smooth] {x^2+1};      
      
      \node[penColor,fill=white] at (axis cs:-.7,.7) {$7$};
      \node[penColor,fill=white] at (axis cs:-1.4,1.4) {$6$};
      \node[penColor,fill=white] at (axis cs:-2.1,2.1) {$5$};
      \node[penColor,fill=white] at (axis cs:-2.8,2.8) {$4$};

      \node[penColor4] at (axis cs:2,4) {$S$};
      \node[penColor] at (axis cs:-3.8,2) {$F$};

      \addplot[penColor2,ultra thick, ->] coordinates{(-1.63,3.65) (-1.22,2.74)};
      \addplot[penColor2,ultra thick, ->] coordinates{(1.63,3.65) (2.04,4.56)};

      \addplot[penColor2,ultra thick, ->] coordinates{(-1.3,2.7) (-.87,1.8)};
      \addplot[penColor2,ultra thick, ->] coordinates{(1.3,2.7) (1.7,3.6)};
      \addplot[penColor2,ultra thick, ->] coordinates{(-.9,1.8) (-.45,.91)};
      \addplot[penColor2,ultra thick, ->] coordinates{(.9,1.8) (1.35,2.7)};
      \addplot[penColor2,ultra thick, ->] coordinates{(0,1) (0,0)};
    \end{axis}
  \end{tikzpicture}
\end{image}

Now we can observe something interesting: If for some point $(x,y)$
the gradient $\grad F(x,y)$ points in the ``general'' direction of
$S$, then $(x,y)$ \textbf{cannot} give an extremal value of $F$, as
moving along $S$ will either increase or decrease the value of
$F$. Here's the upshot:
\begin{quote}
  \textbf{The only candidates for local extrema occur when the
    gradient of $F$ is perpendicular to $S$.}
\end{quote}
How do we find these points? Now we will imagine that $S$ is a level
curve for some other function $G:\R^2\to\R$, and define $S$ as:
\[
S = \{(x,y):G(x,y)= c\}
\]
now, the candidates for extrema of $F$ when constrained to a curve $S$
are found by finding $(x,y)$ such that
\[
\grad F(x,y) =  \lambda \cdot \grad G(x,y)
\]
This is the essence of the method of Lagrange multipliers.

\begin{theorem}[Lagrange Multipliers]
  Let $F:\R^n\to\R$, $G:\R^n\to\R$, $\grad G(\vec{x}) \ne \vec{0}$,
  and let $S$ be the level set
  \[
  S = \{\vec{x}: G(\vec{x}) = c\}
  \]
  If $F$ has extrema when constrained to $S$ at $\vec{x}$, then
  \[
  \grad F(\vec{x}) = \lambda \cdot \grad G(\vec{x})
  \]
  for some number $\lambda$.
\end{theorem}


\section{Working with geometry}


1) Draw and label level curves in mathematica
2) Provide a grid.
3) add hand-drawn curve inkscape.
4) Ask for coordinates of max/min value.


\section{Working with algebra}




\section{The Lagrangian}

Lagrangian.
\[
\mathcal{L}(\vec{x},\lambda)
\]



For some interesting extra reading check out:
\begin{itemize}
\item \link[\textit{Lagrange Multipliers Can Fail to Determine
    Extrema}, J.\ Nunemacher, College Math Journal, January
  2003]{http://www.jstor.org/stable/3595848}.
\item \link[\textit{An ``Extremely'' Cautionary Tale}, M.\ Krusemeyer, College Math Journal, March 2000]{http://www.jstor.org/stable/2687586}.
\item \link[\textit{Unifying a Family of Extrema Problems},
  W.\ Barnier and D.\ Martin, College Math Journal, November
  1997]{http://www.jstor.org/stable/2687071}.
\end{itemize}







\end{document}






\end{document}
