\documentclass{ximera}

\input{../preamble.tex}

\title[Dig-In:]{Line integrals}

\begin{document}
\begin{abstract}
We accumulate vectors along a path.
\end{abstract}
\maketitle

In this section we introduce a new type of integrals, \textit{line integrals} also known as \textit{path integrals}.

\section{Line integrals}

\begin{definition}
Let $\vec{F}:\R^2\to\R^2$ be a vector field, $\vec{p}:\R\to\R^2$ be a
vector valued function,
\begin{align*}
  \vec{F} &= \vector{F_x(x,y), F_y(x,y)}\\
  \vec{p} &= \vector{p_x(t),p_y(t)}.
\end{align*}
\[
\int_C \vec{F}\dotp \d \vec{p} = \int_C \vector{F_x,F_y}\dotp\vector{\d p_x,\d p_y}
\]
Now noting that $\d x = p_x'(t)\d t$ and $\d y = p_y'(t)\d t$  
\begin{align*}
  &= \int_C F_x\cdot \d p_x + F_y\cdot \d p_y\\
  &= \int_C F_x(p_x(t))\cdot p'_x(t) \d t + F_y(p_y(t))\cdot  p'_y(t) \d t\\
  &= \int_C F_x\cdot \d x  + F_y\cdot  \d y
\end{align*}
\end{definition}



\section{The fundamental theorems of calculus}



We now come to the first of three important theorems that extend the
Fundamental Theorem of Calculus to higher dimensions. (The Fundamental
Theorem of Line Integrals has already done this in one way, but in
that case we were still dealing with an essentially one-dimensional
integral.) They all share with the Fundamental Theorem the following
rather vague description: To compute a certain sort of integral over a
region, we may do a computation on the boundary of the region that
involves one fewer integrations.



\[
\int_C \grad F\dotp \d \vec{p} = F(\vec{b}) - F(\vec{a})
\]


\end{document}
