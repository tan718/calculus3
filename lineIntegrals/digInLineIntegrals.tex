\documentclass{ximera}

\input{../preamble.tex}

\title[Dig-In:]{Line integrals}

\begin{document}
\begin{abstract}
We accumulate vectors along a path.
\end{abstract}
\maketitle

In this section we introduce a new type of integrals, \textit{line integrals} also known as \textit{path integrals}.

\section{Line integrals}

A \textit{line integral} measures the flow of a vector field along a
path. 


\begin{definition}
Let $\vec{F}:\R^2\to\R^2$ be a vector field, $\vec{p}:\R\to\R^2$ be a
vector valued function,
\begin{align*}
  \vec{F}(x,y) &= \vector{M(x,y), N(x,y)}\\
  \vec{p}(t) &= \vector{x(t),y(t)}.
\end{align*}
A \dfn{line integral} is an integral of the form:
\[
\int_C \vec{F}\dotp \d \vec{p} = \int_C \vector{M,N}\dotp\vector{\d x,\d y}
\]
Since $\d x = x'(t)\d t$ and $\d y = y'(t)\d t$, we may write  
\begin{align*}
  &= \int_C M\cdot \d x + N\cdot \d y\\
  &= \int_C M(x(t))\cdot x'(t) \d t + N(y(t))\cdot  y'(t) \d t\\
  &= \int_C M\cdot \d x  + N\cdot  \d y
\end{align*}
\end{definition}

\begin{question}
  Consider the following vector field along with a (directed) curve
  $C$.
  \begin{image}
    \begin{tikzpicture}
      \begin{axis}%
        [hide axis,
	  ymin=-4,ymax=4,
	  xmin=-6,xmax=5.5,
	]
        \addplot[penColor,thick, ->] coordinates{(-6,2) (-.5,2)};
        \addplot[penColor,thick, ->] coordinates{(0,2) (5.5,2)};

        \addplot[penColor,thick, ->] coordinates{(-6,1) (-2.5,1)};
        \addplot[penColor,thick, ->] coordinates{(-2,1) (1.5,1)};
        \addplot[penColor,thick, ->] coordinates{(2,1) (5.5,1)};

        \addplot[penColor,thick, ->] coordinates{(-6,0) (-3.5,0)};
        \addplot[penColor,thick, ->] coordinates{(-3,0) (-.5,0)};
        \addplot[penColor,thick, ->] coordinates{(0,0) (2.5,0)};
        \addplot[penColor,thick, ->] coordinates{(3,0) (5.5,0)};

        \addplot[penColor,thick, ->] coordinates{(-6,-1) (-4.5,-1)};
        \addplot[penColor,thick, ->] coordinates{(-4,-1) (-2.5,-1)};
        \addplot[penColor,thick, ->] coordinates{(-2,-1) (-.5,-1)};
        \addplot[penColor,thick, ->] coordinates{(0,-1) (1.5,-1)};
        \addplot[penColor,thick, ->] coordinates{(2,-1) (3.5,-1)};
        \addplot[penColor,thick, ->] coordinates{(4,-1) (5.5,-1)};
        
        \addplot[penColor,thick, ->] coordinates{(-6,-2) (-5.5,-2)};
        \addplot[penColor,thick, ->] coordinates{(-5,-2) (-4.5,-2)};
        \addplot[penColor,thick, ->] coordinates{(-4,-2) (-3.5,-2)};
        \addplot[penColor,thick, ->] coordinates{(-3,-2) (-2.5,-2)};
        \addplot[penColor,thick, ->] coordinates{(-2,-2) (-1.5,-2)};
        \addplot[penColor,thick, ->] coordinates{(-1,-2) (-.5,-2)};
        \addplot[penColor,thick, ->] coordinates{(0,-2) (.5,-2)};
        \addplot[penColor,thick, ->] coordinates{(1,-2) (1.5,-2)};
        \addplot[penColor,thick, ->] coordinates{(2,-2) (2.5,-2)};
        \addplot[penColor,thick, ->] coordinates{(3,-2) (3.5,-2)};
        \addplot[penColor,thick, ->] coordinates{(4,-2) (4.5,-2)};
        \addplot[penColor,thick, ->] coordinates{(5,-2) (5.5,-2)};
        
        \addplot[penColor2,ultra thick] coordinates{
          (-3,2.3) (3,2.3)
          (3,-2.3) (-3,-2.3)
        };
        \addplot[penColor2,ultra thick, ->] coordinates{(-3,2.3) (0,2.3)};
        \addplot[penColor2,ultra thick, ->] coordinates{(3,2.3) (3,0)};
        \addplot[penColor2,ultra thick, ->] coordinates{(3,-2.3) (0,-2.3)};
      \end{axis}
    \end{tikzpicture}
  \end{image}

  
  Picture see 2153-11-9
\end{question}


\begin{example}
  Let $\vec{F}(x,y) = \vector{-y,x}$ and let $C$ be the unit circle
  centered at the origin. Compute
  \[
  \int_C \vec{F}\dotp d\vec{p}
  \]
  \begin{explanation}
    The path $C$ can be parameterized by
    \begin{align*}
      x(\theta) &= \cos(\theta)\\
      y(\theta) &= \sin(\theta)
    \end{align*}
    with $0\le \theta\le 2\pi$. To compute the integral, write with me
    \begin{align*}
      \int_C \vec{F}\dotp d\vec{p} &= \int_0^{2\pi} F(x(\theta),y(\theta))\dotp \vector{x'(\theta),y'(\theta)}\d \theta\\
      &= \int_0^{2\pi} \vector{-\sin(\theta),\cos(\theta)}\dotp \vector{-\sin(\theta),\cos(\theta)}\d \theta\\
      &= \int_0^{2\pi}\left(\sin^2(\theta)+\cos^2(\theta)\right)\d \theta\\
      &= \int_0^{2\pi} 1\d \theta\\
      &=2\pi.
    \end{align*}
  \end{explanation}
\end{example}



\section{The fundamental theorems of calculus}



We now come to the first of three important theorems that extend the
Fundamental Theorem of Calculus to higher dimensions. (The Fundamental
Theorem of Line Integrals has already done this in one way, but in
that case we were still dealing with an essentially one-dimensional
integral.)

They all share with the Fundamental Theorem the following
rather vague description:

To compute a certain sort of integral over a
region, we may do a computation on the boundary of the region that
involves one fewer integrations.


\begin{theorem}[Fundamental Theorem for Line Integrals]
  If $C$ is a curve that starts at $\vec{a}$ and ends at $\vec{b}$
  \[
  \int_C \grad F\dotp \d \vec{p} = F(\vec{b}) - F(\vec{a})
  \]
\end{theorem}

\begin{question}
\end{question}

\begin{example}
\end{example}



\end{document}
