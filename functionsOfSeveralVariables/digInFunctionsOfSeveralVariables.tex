\documentclass{ximera}

\input{../preamble}

\outcome{}

\title[Dig-In:]{Functions of several variables}

\begin{document}
\begin{abstract}
\end{abstract}
\maketitle


The world is constantly changing. Even mountains are not impervious to
change.  Here we see Mount St.\ Helens on May 17th, 1980:
%% https://commons.wikimedia.org/wiki/File:Mount_St._Helens,_one_day_before_the_devastating_eruption.jpg
%% public domain
\begin{image}
  \includegraphics{MSHbefore.jpg}
\end{image}

Here we see Mount St.\ Helens on May 19, 1982:
%% https://commons.wikimedia.org/wiki/File:MSH82_st_helens_plume_from_harrys_ridge_05-19-82.jpg
%% public domain
\begin{image}
  \includegraphics{MSHafter.jpg}
\end{image}
If we want to really understand how Mount St.\ Helens changed, we
first need a way of describing the change. Cartographers use contour
maps to show how the three-dimensional structure of the mountain changed:
\begin{image}
image with contour lines
\end{image}
In essence we are looking at the mountain from above, and each line in
the maps above represent a fixed, constant height.  In a similar way,
we will describe functions of two (and three!)  variables using
contour lines.


\section{Contour maps}


\section{Quadratic surfaces}




\end{document}
