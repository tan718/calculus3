\documentclass{ximera}

\input{../preamble.tex}


\title[Dig-In:]{Taylor polynomials}

\begin{document}
\begin{abstract}
  We introduce Taylor polynomials for functions of several variables.
\end{abstract}
\maketitle


Recall the definition of a \textit{Taylor polynomial}:

\begin{definition}
  Let $f:\R\to\R$ be a function whose first $d$ derivatives exist at $x=c$.
  The \dfn{Taylor polynomial} of degree $d$ of $f$ centered at $x=c$ is
  \begin{align*}
    p_d(x) = \sum_{k=0}^d\frac{f^{(k)}(c)}{k!}(x-c)^k.
  \end{align*}
\end{definition}

\begin{question}
  Let $f(x) = \sin(x)$. Compute
  \[
  p_7(x)\begin{prompt}
    = \answer{x-x^3/3!+x^5/5!-x^7/7!}
  \end{prompt}
  \]
  \begin{question}
  Let $f(x) = \cos(x)$. Compute
  \[
  p_7(x)\begin{prompt}
    = \answer{1-x^2/2+x^4/4!-x^6/6!}
  \end{prompt}
  \]
  \begin{question}
    Let $f(x) = e^x$. Compute
    \[
    p_7(x)\begin{prompt}
      = \answer{1 + x + x^2/2+ + x^3/3! + x^4/4!+ x^5/5! + x^6/6!+x^7/7!}
    \end{prompt}
    \]
  \end{question}
\end{question}
\end{question}

We have a similar formula for functions $F:\R^n\to \R$:
\begin{definition}
  Let $F:\R^n\to\R$ be a function whose first $d$ derivatives exist at
  $\vec{x}=\vec{c}$.  The \dfn{Taylor polynomial} of degree $d$ of $F$
  centered at $\vec{x}=\vec{c}$ is
  \[
  P_d(\vec{x}) = \sum_{k=0}^d \eval{\eval{\frac{(\vec{a}\dotp \grad)^k F(\vec{x})}{k!}}_{\vec{x}=\vec{c}}}_{\vec{a}=\vec{x}-\vec{c}}
  \]
\end{definition}
Don't panic! This formula is complex and will take some
unpacking. We'll walk you through this.

\subsection{The degree zero Taylor polynomial}
First note that
\begin{align*}
  P_0(\vec(x)) &= \sum_{k=0}^0 \eval{\eval{\frac{(\vec{a}\dotp \grad)^k F(\vec{x})}{k!}}_{\vec{x}=\vec{c}}}_{\vec{a}=\vec{x}-\vec{c}}\\
  &=\eval{\eval{\frac{(\vec{a}\dotp \grad)^0 F(\vec{x})}{0!}}_{\vec{x}=\vec{c}}}_{\vec{a}=\vec{x}-\vec{c}}\\
  &=\eval{\eval{F(\vec{x})}_{\vec{x}=\vec{c}}}_{\vec{a}=\vec{x}-\vec{c}}\\
  &=F(\vec{c}).
\end{align*}
This means for any function $F:\R^n\to\R$, the $0$th degree Taylor
polynomial for $F$ at $\vec{x}=\vec{c}$ is just
\[
P_0(\vec{x})=F(\vec{c}).
\]
\begin{question}
  Consider $F(x,y)= \sin(x+y)$. Compute the $0$th degree Taylor
  polynomial for $F$ at $\vector{x,y} =\vector{\pi/4,\pi/4}$.
  \begin{prompt}
    \[
    P_0(x,y) = \answer{1}
    \]
  \end{prompt}
\end{question}

\subsection{The  degree one Taylor polynomial}
Now let's look at the $1$st degree Taylor polynomial:
\begin{align*}
  P_1(\vec{x})&= \sum_{k=0}^1 \eval{\eval{\frac{(\vec{a}\dotp \grad)^k F(\vec{x})}{k!}}_{\vec{x}=\vec{c}}}_{\vec{a}=\vec{x}-\vec{c}}\\
  &=P_0(\vec{x}) + \eval{\eval{\frac{(\vec{a}\dotp \grad)^1 F(\vec{x})}{1!}}_{\vec{x}=\vec{c}}}_{\vec{a}=\vec{x}-\vec{c}}\\
  &=F(\vec{c}) + \eval{\eval{(\vec{a}\dotp \grad) F(\vec{x})}_{\vec{x}=\vec{c}}}_{\vec{a}=\vec{x}-\vec{c}}
\end{align*}
Now we ask, what is $(\vec{a}\dotp\grad)F$?

Computing the dot product, 
\[
(\vec{a}\dotp\grad) = a_1 \pp{x} + a_2 \pp{y}
\]
and to find $(\vec{a}\dotp\grad)F$, we distribute $F$ to obtain
\begin{align*}
  (\vec{a}\dotp\grad)F &= a_1 \pp[F]{x} + a_2 \pp[F]{y}\\
  &= \grad F(\vec{x})\dotp \vec{a}
\end{align*}
So
\begin{align*}
  P_1(\vec{x})&=F(\vec{c}) + \eval{\eval{\grad F(\vec{x})\dotp{a}}_{\vec{x}=\vec{c}}}_{\vec{a}=\vec{x}-\vec{c}}\\
  &=F(\vec{c}) +\grad F(\vec{c})\dotp (\vec{x}-\vec{c}).
\end{align*}
This means for any function $F:\R^n\to\R$, the $1$st degree Taylor
polynomial for $F$ at $\vec{x}=\vec{c}$ is just the tangent ``plane''
for $F$ at $\vec{x}= \vec{c}$.


\begin{question}
  Consider $F(x,y)= 3+4x-5y$. Compute the $1$st degree Taylor
  polynomial for $F$ at $\vector{x,y} =\vector{0,0}$.
  \begin{prompt}
    \[
    P_1(x,y) = \answer{3+4x-5y}
    \]
  \end{prompt}
\end{question}

\subsection{The  degree two Taylor polynomial}
To get our hands on the $2$nd degree Taylor polynomial, we will
specialize to functions $F:\R^2\to\R$. Let $\vec{c}=\vector{c_1,c_2}$
and let $\vec{x} = \vector{x,y}$.  Write with me:
\begin{align*}
  P_2(\vec{x})&= P_1(\vec{x}) + \eval{\eval{\frac{(\vec{a}\dotp \grad)^2 F(\vec{x})}{2!}}_{\vec{x}=\vec{c}}}_{\vec{a}=\vec{x}-\vec{c}}\\
  &= F(\vec{c})
+ \grad F(\vec{c})\dotp (\vec{x}-\vec{c})
+(1/2)\eval{\eval{(\vec{a}\dotp \grad)^2 F(\vec{x})}_{\vec{x}=\vec{c}}}_{\vec{a}=\vec{x}-\vec{c}}
\end{align*}
Now we ask ourselves, what is $(\vec{a}\dotp \grad)^2 F(\vec{x})$?

We know that
\[
(\vec{a}\dotp\grad)F = a_1 \pp[F]{x} + a_2 \pp[F]{y}
\]
and 
\begin{align*}
(\vec{a}\dotp\grad)^2F &=(\vec{a}\dotp\grad)(\vec{a}\dotp\grad)F\\
  &= (\vec{a}\dotp\grad)\left(a_1 \pp[F]{x} + a_2 \pp[F]{y}\right)\\
  &= \left(a_1 \pp[F]{x} + a_2 \pp[F]{y}\right)\left(a_1 \pp[F]{x} + a_2 \pp[F]{y}\right)
\end{align*}
Now we use the distributivity property and (since we are assuming that
all derivatives of $F$ exist) we have that
\[
\frac{\partial^2F}{\partial x\partial y}  = \frac{\partial^2F}{\partial y\partial x}
\]
so
\[
(\vec{a}\dotp\grad)^2F = a_1^2\frac{\partial^2F}{\partial x^2} + 2a_1
a_2\frac{\partial^2F}{\partial x\partial y} +
a_2^2\frac{\partial^2F}{\partial y^2}.
\]
We can now unpack our second degree Taylor polynomial:
\begin{align*}
  P_2(\vec{x})= F(\vec{c})
&+ \grad F(\vec{c})\dotp (\vec{x}-\vec{c})\\
&+(1/2)\eval{\eval{(\vec{a}\dotp\grad)^2F}_{\vec{x}=\vec{c}}}_{\vec{a}=\vec{x}-\vec{c}},
\end{align*}
as
\begin{align*}
P_2(\vec{x})=F(\vec{c})
&+ \grad F(\vec{c})\dotp (\vec{x}-\vec{c})\\
&+(1/2)\eval{\eval{a_1^2\frac{\partial^2F}{\partial x^2} + 2a_1 
a_2\frac{\partial^2F}{\partial x\partial y} +
a_2^2\frac{\partial^2F}{\partial y^2}}_{\vec{x}=\vec{c}}}_{\vec{a}=\vec{x}-\vec{c}}.
\end{align*}
and finally we see:
\begin{align*}
P_2(\vec{x})=F(\vec{c})
&+ \grad F(\vec{c})\dotp (\vec{x}-\vec{c})\\
&+\frac{F^{(2,0)}(\vec{c}) (x-c_1)^2 + 2F^{(1,1)}(\vec{c})(x-c_1)(y-c_2)  + F^{(0,2)}(\vec{c})(y-c_2)^2}{2}
\end{align*}


\section{Try it, you might like it}

Computing the Taylor polynomial is not so bad, you just need to get the hang of it. 

\begin{example}
  Compute the degree $2$ Taylor polynomial for:
  \[
  F(x,y)=\sin(xy)
  \]
  centered at $(0,0)$.
  \begin{explanation}
    We'll start by making a table of parital derivatives along with
    their value when evaluated at $(0,0)$:
    \[
    \begin{array}{lcl}
      F(x,y) = \cos(xy) & \Rightarrow &F(0,0) = 0\\
      F^{(1,0)}(x,y) = \answer[given]{y\cos(xy)} & \Rightarrow & F^{(1,0)}(0,0) = \answer[given]{0}\\
      F^{(0,1)}(x,y) = \answer[given]{x\cos(xy)} &\Rightarrow  & F^{(0,1)}(0,0) = \answer[given]{0}\\
      F^{(2,0)}(x,y) = \answer[given]{-y^2\sin(xy)} &\Rightarrow & F^{(2,0)}(0,0) = \answer[given]{0}\\
      F^{(0,2)}(x,y) = \answer[given]{-x^2\sin(xy)} &\Rightarrow & F^{(0,2)}(0,0) = \answer[given]{0}\\
      F^{(1,1)}(x,y) = \answer[given]{\cos(xy)-xy\sin(xy)} &\Rightarrow & F^{(1,1)}(0,0) = \answer[given]{1}
    \end{array}
    \]
    We see our polynomial is just
    \begin{align*}
      P_2(x,y) &= \answer{given}{1}\cdot (x-0)\cdot (y-0)\\
      &= \answer[given]{xy}.
    \end{align*}
  \end{explanation}
\end{example}


\begin{example}
  Compute the degree $2$ Taylor polynomial for:
  \[
  F(x,y)= x^2 y + y^2 + x y
  \]
  centered at $(-1,0)$.
  \begin{explanation}
    We'll start by making a table of parital derivatives along with
    their value when evaluated at $(-1,0)$:
    \[
    \begin{array}{lcl}
      F(x,y) = x^2y+y^2+xy & \Rightarrow &F(-1,0) = 0\\
      F^{(1,0)}(x,y) = \answer[given]{2xy+y} & \Rightarrow & F^{(1,0)}(-1,0) = \answer[given]{0}\\
      F^{(0,1)}(x,y) = \answer[given]{x^2+2y+x} &\Rightarrow  & F^{(0,1)}(-1,0) = \answer[given]{0}\\
      F^{(2,0)}(x,y) = \answer[given]{2y} &\Rightarrow & F^{(2,0)}(-1,0) = \answer[given]{0}\\
      F^{(0,2)}(x,y) = \answer[given]{2} &\Rightarrow & F^{(0,2)}(-1,0) = \answer[given]{2}\\
      F^{(1,1)}(x,y) = \answer[given]{1+2x} &\Rightarrow & F^{(1,1)}(-1,0) = \answer[given]{-1}
    \end{array}
    \]
    We see our polynomial is
    \begin{align*}
      P_2(x,y) &= (1/2)\left(\answer[given]{2}\cdot (y-0)^2 + 2 (\answer[given]{-1})(x-(-1))(y-0) \right)\\
      &= \answer[given]{y^2-(x+1)y}.
    \end{align*}
  \end{explanation}
\end{example}


\begin{example}
  Compute the degree $2$ Taylor polynomial for:
  \[
  F(x,y)= x^3-3x-y^2+4y
  \]
  centered at $(-1,2)$.
  \begin{explanation}
    We'll start by making a table of parital derivatives along with
    their value when evaluated at $(-1,2)$:
    \[
    \begin{array}{lcl}
      F(x,y) = x^3-3x-y^2+4y & \Rightarrow &F(-1,2) = 6\\
      F^{(1,0)}(x,y) = \answer[given]{3x^2-3} & \Rightarrow & F^{(1,0)}(-1,2) = \answer[given]{0}\\
      F^{(0,1)}(x,y) = \answer[given]{-2y+4} &\Rightarrow  & F^{(0,1)}(-1,2) = \answer[given]{0}\\
      F^{(2,0)}(x,y) = \answer[given]{6x} &\Rightarrow & F^{(2,0)}(-1,2) = \answer[given]{-6}\\
      F^{(0,2)}(x,y) = \answer[given]{-2} &\Rightarrow & F^{(0,2)}(-1,2) = \answer[given]{-2}\\
      F^{(1,1)}(x,y) = \answer[given]{0} &\Rightarrow & F^{(1,1)}(-1,2) = \answer[given]{0}
    \end{array}
    \]
    We see our polynomial is
    \begin{align*}
      P_2(x,y) &= 6 + (1/2)\left(\answer[given]{-6}\cdot (x-(-1))^2 +  (\answer[given]{-2})(y-2)^2 \right)\\
      &= \answer[given]{6-3(x+1)^2-2(y-2)^2}.
    \end{align*}
  \end{explanation}
\end{example}

\begin{question}
  Compute the degree $2$ Taylor polynomial for:
  \[
  F(x,y)=\cos(xy)
  \]
  centered at $(0,0)$.
  \begin{hint}
    Start by making a table of parital derivatives along with
    their value when evaluated at $(0,0)$.
  \end{hint}
    \begin{prompt}
    \[
    P_2(x,y) = \answer{1}
    \]
  \end{prompt}
\end{question}

\begin{question}
  Compute the degree $2$ Taylor polynomial for:
  \[
  F(x,y)= x^2 y + y^2 + x y
  \]
  centered at $(-1/2,1/8)$.
  \begin{hint}
    Start by making a table of parital derivatives along with
    their value when evaluated at $(-1/2,1/8)$.
  \end{hint}
  \begin{prompt}
    \[
    P_2(x,y) = \answer{-1/64 + (1/8)(x+1/2)^2+(y-1/8)^2}
    \]
  \end{prompt}
\end{question}

\begin{question}
  Compute the degree $2$ Taylor polynomial for:
  \[
  F(x,y)= x^3-3x-y^2+4y
  \]
  centered at $(1,2)$.
  \begin{hint}
    Start by making a table of parital derivatives along with
    their value when evaluated at $(1,2)$.
  \end{hint}
  \begin{prompt}
    \[
    P_2(x,y) = \answer{2 + 3(x-1)^2 -(y-2)^2}
    \]
  \end{prompt}
\end{question}


\end{document}
