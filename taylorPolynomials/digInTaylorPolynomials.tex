\documentclass{ximera}

\input{../preamble.tex}


\title[Dig-In:]{Taylor polynomials}

\begin{document}
\begin{abstract}
  We introduce Taylor polynomials for functions of several variables.
\end{abstract}
\maketitle


Recall the definition of a \textit{Taylor polynomial}:

\begin{definition}
  Let $f$ be a function whose first $d$ derivatives exist at $x=c$.
  The \dfn{Taylor polynomial} of degree $d$ of $f$ at $x=c$ is
  \begin{align*}
    p_d(x) = \sum_{k=0}^d\frac{f^{(k)}(c)}{k!}(x-c)^k.
  \end{align*}
\end{definition}

\begin{question}
  Let $f(x) = \sin(x)$. Compute
  \[
  p_7(x)\begin{prompt}
    = \answer{x-x^3/3!+x^5/5!-x^7/7!}
  \end{prompt}
  \]
  \begin{question}
  Let $f(x) = \cos(x)$. Compute
  \[
  p_7(x)\begin{prompt}
    = \answer{1-x^2/2+x^4/4!-x^6/6!}
  \end{prompt}
  \]
  \begin{question}
    Let $f(x) = e^x$. Compute
    \[
    p_7(x)\begin{prompt}
      = \answer{1 + x + x^2/2+ + x^3/3! + x^4/4!+ x^5/5! + x^6/6!+x^7/7!}
    \end{prompt}
    \]
  \end{question}
\end{question}
\end{question}



\[
P_d(\vec{x}) = \sum_{k=0}^d \eval{\eval{\frac{(\vec{a}\dotp \grad)^k F(\vec{x})}{k!}}_{\vec{x}=\vec{c}}}_{\vec{a}=\vec{x}-\vec{c}}
\]

This will take some unpacking. Recall that tangent planes are given by

\[
P_1(\vec{x}) = F(\vec{c}) + \grad F(\vec{c})\dotp (\vec{x}-\vec{c}).
\]


\[
\grad = \vector{\pp{x_1},\pp{x_2},\dots,\pp{x_n}}
\]
\[
\vec{a} = \vector{a_1,a_2,\dots,a_n}
\]
\[
\vec{a}\dotp\grad=a_1\pp{x_1}+a_2\pp{x_2}+\dots+a_n\pp{x_n}
\]


\section{Working in three dimensions}

We'll mostly be working with functions of the form: $F:\R^2\to
\R$. Recall that these can be viewed as surfaces in $\R^3$.

\[
(\vec{a}\dotp\grad)F = a_1 \pp[F]{x} + a_2 \pp[F]{y}
\]

\begin{align*}
(\vec{a}\dotp\grad)^2F &=(\vec{a}\dotp\grad)(\vec{a}\dotp\grad)F\\
  &= (\vec{a}\dotp\grad)\left(a_1 \pp[F]{x} + a_2 \pp[F]{y}\right)\\
  &= \left(a_1 \pp[F]{x} + a_2 \pp[F]{y}\right)\left(a_1 \pp[F]{x} + a_2 \pp[F]{y}\right)
\end{align*}
Now we use the distributivity property and since we are assuming that
all derivatives of $F$ exist, we have that
\[
\frac{\partial^2F}{\partial x\partial y}  = \frac{\partial^2F}{\partial y\partial x}
\]
so
\[
(\vec{a}\dotp\grad)^2F = a_1^2\frac{\partial^2F}{\partial x^2} + 2a_1
a_2\frac{\partial^2F}{\partial x\partial y} +
a_2^2\frac{\partial^2F}{\partial y^2}.
\]
In an entirely similar way, we could compute more of the polynomail,
but we won't do that here.

\section{Try it, you might like it}

Examples:

\[
\sin(x+ y)
\]

\begin{example}
  Compute the Taylor series for $f(x) = \sin(x)$ centered at $x=0$.
  \begin{explanation}
    We'll start by making a table of derivatives:
    \[
    \begin{array}{lcl}
      f(x) = \sin(x) & \Rightarrow &f(0) = 0\\
      f'(x) = \answer[given]{\cos(x)} & \Rightarrow & f'(0) = \answer[given]{1}\\
      f''(x) = \answer[given]{-\sin(x)} &\Rightarrow &f''(0) = \answer[given]{0}\\
      f'''(x) = \answer[given]{-\cos(x)} &\Rightarrow &f'''(0) = \answer[given]{-1}\\
      f^{(4)}(x) = \answer[given]{\sin(x)} &\Rightarrow &f^{(4)}(0) = \answer[given]{0}\\
      f^{(5)}(x) = \answer[given]{\cos(x)} &\Rightarrow &f^{(5)}(0) = \answer[given]{1}\\
      f^{(6)}(x) = \answer[given]{-\sin(x)} &\Rightarrow &f^{(6)}(0) =\answer[given]{0}\\
      f^{(7)}(x) = \answer[given]{-\cos(x)} &\Rightarrow &f^{(7)}(0) = \answer[given]{-1}\\
      f^{(8)}(x) = \answer[given]{\sin(x)} &\Rightarrow &f^{(8)}(0) = \answer[given]{0}\\
      f^{(9)}(x) = \answer[given]{\cos(x)} &\Rightarrow &f^{(9)}(0) = \answer[given]{1}\\
    \end{array}
    \]
    Since a repeating pattern has emerged, we see that the Maclaurin
    series for $\sin(x)$ is:
    \[
    x-\frac{x^3}{3!}+\frac{x^5}{5!}-\frac{x^7}{7!}+\cdots = \sum_{n=0}^\infty \frac{(-1)^{n+1}}{(2n+1)!} x^{2n+1}
    \]
  \end{explanation}
\end{example}

\end{document}
