\documentclass{ximera}

\input{../preamble.tex}

\title[Dig-In:]{Introduction to multiple integrals}

\begin{document}
\begin{abstract}
NO TITLE YET.
\end{abstract}
\maketitle


%% Now we move on to finding \textit{volumes} under surfaces. We start by
%% only considering simple regions, by this we mean rectangles and boxes.

%% \section{Integrals over trivial regions}

%% We will first discuss \textit{double integrals}, and next discuss
%% \textit{triple integrals}.

%% \subsection{Double integrals}

%% Suppose you have a function $F:\R^2\to\R$. A graph of this function is
%% a surface in $\R^3$. For example:
%% \begin{image}
%%   \begin{tikzpicture}
%%     \begin{axis}%
%%       [ tick label style={font=\scriptsize},axis on top,
%% 	axis lines=center,
%% 	view={110}{25},
%% 	name=myplot,
%% 	xtick=\empty,
%%         ytick=\empty,
%%         ztick=\empty,
%% 	ymin=-1.2,ymax=1.2,
%% 	xmin=-.5,xmax=2.5,
%% 	zmin=-.1, zmax=2.1,
%% 	every axis x label/.style={at={(axis cs:\pgfkeysvalueof{/pgfplots/xmax},0,0)},xshift=-1pt,yshift=-4pt},
%% 	xlabel={\scriptsize $x$},
%% 	every axis y label/.style={at={(axis cs:0,\pgfkeysvalueof{/pgfplots/ymax},0)},xshift=5pt,yshift=-3pt},
%% 	ylabel={\scriptsize $y$},
%% 	every axis z label/.style={at={(axis cs:0,0,\pgfkeysvalueof{/pgfplots/zmax})},xshift=0pt,yshift=4pt},
%% 	zlabel={\scriptsize $z$},
%%         colormap/cool,
%%       ]
%%       %% Surface %% 12 x 16
%%       \addplot3[domain=0:2,y domain=-1:1,mesh,samples=13,samples y=17,very thin,z buffer=sort] {-.5*(x-1)^2-.5*(y)^2+2};
%%     \end{axis}
%%   \end{tikzpicture}
%% \end{image}

%% If we want to compute the signed volume (we use the term ``signed
%% volume'' to denote that space above the $(x,y)$ plane, under $F$, will
%% have a positive volume; space above $F$ and under the $(x,y)$-plane
%% will have a ``negative'' volume, similar to the notion of signed area
%% used before) of this surface over a rectangular region, say the
%% rectangle defined by
%% \[
%% R = \{(x,y):\text{$a\le x\le b$ and $c\le y\le d$}\}
%% \]
%% we break $R$ into $n$ slices parallel to the $x$-axis, and $m$ slices
%% parallel to the $y$-axis. This allows us to consider boxes of
%% dimension
%% \[
%% F(x_i^*,y_j^*)\times\underbrace{\left(\frac{b-a}{m}\right)}_{\Delta x}\times \underbrace{\left(\frac{d-c}{n}\right)}_{\Delta y}
%% \]
%% where $(x_i^*,y_j^*)$ is a point in $(i,j)$-rectangle:
%% \begin{image}
%%   \begin{tikzpicture}
%%     \begin{axis}%
%%       [tick label style={font=\scriptsize},axis on top,
%% 	axis lines=center,
%% 	view={110}{25},
%% 	name=myplot,
%%         xtick=\empty,
%%         ytick=\empty,
%%         ztick=\empty,
%% 	ymin=-1.2,ymax=1.2,
%% 	xmin=-.5,xmax=2.5,
%% 	zmin=-.1, zmax=2.1,
%% 	every axis x label/.style={at={(axis cs:\pgfkeysvalueof{/pgfplots/xmax},0,0)},xshift=-1pt,yshift=-4pt},
%% 	xlabel={\scriptsize $x$},
%% 	every axis y label/.style={at={(axis cs:0,\pgfkeysvalueof{/pgfplots/ymax},0)},xshift=5pt,yshift=-3pt},
%% 	ylabel={\scriptsize $y$},
%% 	every axis z label/.style={at={(axis cs:0,0,\pgfkeysvalueof{/pgfplots/zmax})},xshift=0pt,yshift=4pt},
%% 	zlabel={\scriptsize $z$},
%%         colormap/cool,clip=false,
%%       ]
%%       %% 12 lines
%%       \pgfplotsinvokeforeach{0,0.1667,...,2}{
%%         \draw[gray] (axis cs: #1,-1,0) -- (axis cs: #1 , 1,0);
%%       }

%%       %% 16 lines
%%       \pgfplotsinvokeforeach{-1,-.875,...,1}{
%%         \draw[gray] (axis cs: 0,#1,0) -- (axis cs: 2 ,#1,0);
%%       }


%%       %% \pgfplotsinvokeforeach{-1,-.875,...,1}{
%%       %%   \addplot3[domain=0:2,y domain=-1:1,mesh,samples=13,samples
%%       %%     y=17,very thin,z buffer=sort] {-.5*(x-1)^2-.5*(y)^2+2};
%%       %%   }

      

%%       %% Box vol
%%       \draw [gray,thick]
%%       (axis cs: 1.833,.25,0) --
%%       (axis cs: 1.833,.375,0) --
%%       (axis cs: 1.833,.375,1.665) --
%%       (axis cs: 1.833,.25,1.665) --
%%       (axis cs: 1.833,.25,0);

%%       \draw [gray,thick]
%%       (axis cs: 1.833,.375,0) --
%%       (axis cs: 1.833,.375,1.665) --
%%       (axis cs: 1.667,.375,1.665) --
%%       (axis cs: 1.667,.375,0) --
%%       (axis cs: 1.833,.375,0);

%%       \draw [gray,thick]
%%       (axis cs: 1.667,.375,1.665) --
%%       (axis cs: 1.667,.25,1.665) --
%%       (axis cs: 1.833,.25,1.665);

%%       %% Surface %% 12 x 16
%%       \addplot3[domain=0:2,y domain=-1:1,mesh,samples=13,samples y=17,very thin,z buffer=sort] {-.5*(x-1)^2-.5*(y)^2+2};

%%       %% Surface curves
%%       \addplot3[domain=0:2,%fill=white,
%%         penColor,very thick,samples=13,samples y=0] (
%%                {x},
%%                {1},
%%                {-.5*(x-1)^2-.5*(1)^2+2});

%%       \addplot3[domain=0:.5,%fill=white,
%%         penColor,very thick,dashed,samples=13,samples y=0] (
%%                {x},
%%                {-1},
%%                {-.5*(x-1)^2-.5*(-1)^2+2});

%%       \addplot3[domain=.5:2,%fill=white,
%%         penColor,very thick,samples=13,samples y=0] (
%%                {x},
%%                {-1},
%%                {-.5*(x-1)^2-.5*(-1)^2+2});

%%       \addplot3[domain=-1:1,%fill=white,
%%         penColor,very thick,samples=13,samples y=18] (
%%                {2},
%%                {y},
%%                {-.5*(2-1)^2-.5*(y)^2+2});

%%       \addplot3[domain=-1:.8,%fill=white,
%%         dashed,penColor,very thick,samples=13,samples y=18] (
%%                {0},
%%                {y},
%%                {-.5*(0-1)^2-.5*(y)^2+2});

%%       \addplot3[domain=.75:1,%fill=white,
%%         penColor,very thick,samples=13,samples y=18] (
%%                {0},
%%                {y},
%%                {-.5*(0-1)^2-.5*(y)^2+2});

%%       %% %% dxdydz         
%%       %% \draw [penColor, thick]
%%       %% (axis cs: 1.667,.25,1.746) --(axis cs: 1.667,.375,1.71) --
%%       %% (axis cs: 1.833,.375,1.583) --(axis cs: 1.833,.25,1.621) --
%%       %% (axis cs: 1.667,.25,1.746) -- (axis cs: 1.667,.375,1.71);
      
%%       \draw[decoration={brace,raise=.1cm},decorate,thin] (axis cs:0,-1,0)--(axis cs: 0,1,0);
%%       \node at (axis cs: -.6,-.1,0) {\scriptsize$n$};
      
%%       \draw[decoration={brace,mirror,raise=.1cm},decorate,thin] (axis cs:0,-1,0)--(axis cs: 2,-1,0);
%%       \node at (axis cs: .7,-1.25,0) {\scriptsize$m$};
      
%%       \draw[->,thin] (axis cs:1.75,1.2,0)--(axis cs: 1.75,.375,0);
%%       \node at (axis cs: 1.75,1.3,0) {\scriptsize$\Delta x$};

%%       \draw[->,thin] (axis cs:2.2,.3125,0)--(axis cs: 1.833,.3125,0);
%%       \node at (axis cs: 2.4,.3125,0) {\scriptsize$\Delta y$};

%%       \draw[decoration={brace,mirror,raise=.1cm},decorate,thin] (axis cs: 1.833,.25,1.665) -- (axis cs: 1.833,.25,0);
%%       \node[above left] at (axis cs: 1.833,.2,.8) {\scriptsize$ F(x_i^*,y_j^*)$};
%%     \end{axis}
%%   \end{tikzpicture}
%% \end{image}

%% Computing the volume of each of these boxes approximates the signed volume enclosed by the surface: 

%% \begin{image}
%% \begin{tikzpicture}
%%     \begin{axis}%
%%       [tick label style={font=\scriptsize},axis on top,
%% 	axis lines=center,
%% 	view={110}{25},
%% 	name=myplot,
%% 	xtick=\empty,
%% 	ytick=\empty,
%%      ztick=\empty,
%% 	ymin=-1.2,ymax=1.2,
%% 	xmin=-.5,xmax=2.5,
%% 	zmin=-.1, zmax=2.1,
%% 	every axis x label/.style={at={(axis cs:\pgfkeysvalueof{/pgfplots/xmax},0,0)},xshift=-1pt,yshift=-4pt},
%% 	xlabel={\scriptsize $x$},
%% 	every axis y label/.style={at={(axis cs:0,\pgfkeysvalueof{/pgfplots/ymax},0)},xshift=5pt,yshift=-3pt},
%% 	ylabel={\scriptsize $y$},
%% 	every axis z label/.style={at={(axis cs:0,0,\pgfkeysvalueof{/pgfplots/zmax})},xshift=0pt,yshift=4pt},
%% 	zlabel={\scriptsize $z$},
%%         colormap/cool,
%%       ]
%%       \foreach \j in {0,.125,...,1.875}{
%%        \foreach \i in {0,0.1667,...,1.8333}{        
%%         %% RIGHT SIDE
%%         \addplot3[opacity=1,surf,domain=0+\i:.1667+\i,y domain=0:{-.5*(((0+\i +.1667+\i )/2)-1)^2-.5*((-1+\j-.875+\j)/2)^2+2},samples=2,samples y=2,very thin,z buffer=sort] (x,-.875+\j,y);
%%         %% FRONT
%%         \addplot3[opacity=1,surf,domain=-1+\j:-.875+\j,y domain=0:{-.5*(((0+\i +.1667+\i )/2)-1)^2-.5*((-1+\j-.875+\j)/2)^2+2},samples=2,samples y=2,very thin,z buffer=sort] (.1667+\i,x,y);
%%         %% TOP
%%         \addplot3[opacity=1,surf,domain=0+\i:.1667+\i,y domain=-1+\j:-.875+\j,samples=2,samples y=2,very thin,z buffer=sort] (x,y, {-.5*(((0+\i +.1667+\i )/2)-1)^2-.5*((-1+\j-.875+\j)/2)^2+2});
%%       }}
%%     \end{axis}
%% \end{tikzpicture}
%% \end{image}

%% Letting the number of rectangles in the $x$-direction and
%% $y$-direction go to infinity, we will have that
%% \[
%% \Delta x \cdot \Delta y
%% \]
%% goes to zero, and we will find the exact volume enclosed by our surface when
%% bounded by the region $R$. This leads to our definition of a \textit{double integral}:

%% \begin{definition}
%%   Given a function $F:\R^2\to\R$, a \dfn{double integral}
%%   \[
%%   \int_R F(x,y) \d A 
%%   \]
%%   of a function $F$ over a rectangular region $R$, is given by:
%%   \[
%%   \int_R F(x,y) \d A = \lim_{\substack{m\to\infty\\ n\to\infty}}\sum_{i=1}^m\sum_{j=1}^n F(x^*_i,y^*_j)\Delta x\Delta y
%%   \]
%% \end{definition}


%% \begin{question}
%%   Let the value of a function $F:\R^2\to\R$ be given below:
%%   \begin{image}
%%     \begin{tikzpicture}
%%       \begin{axis}
%% 	[xmin=-0.2,
%%           xmax=4.2,
%%           ymin=-0.2,
%%           ymax=6.2,
%%           axis lines=center,
%%           xlabel=$x$,ylabel=$y$,
%%           every axis y label/.style={at=(current axis.above origin),anchor=south},
%%           every axis x label/.style={at=(current axis.right of origin),anchor=west},
%% 	  domain=-1:2,
%%           clip=false,
%% 	  ytick={2,4,6},
%% 	  yticklabels={$2$,$4$,$6$},
%% 	  xtick={1,2,3,4},
%% 	  xticklabels={$1$,$2$,$3$,$4$},
%% 	  grid = major
%% 	]
%% 	\node at (axis cs: .5,1) {$1$};
%%         \node at (axis cs: .5,3) {$-2$};
%%         \node at (axis cs: .5,5) {$-4$};
        
%%         \node at (axis cs: 1.5,1) {$2$};
%%         \node at (axis cs: 1.5,3) {$0$};
%%         \node at (axis cs: 1.5,5) {$-1$};
        
%%         \node at (axis cs: 2.5,1) {$5$};
%%         \node at (axis cs: 2.5,3) {$3$};
%%         \node at (axis cs: 2.5,5) {$2$};
        
%%         \node at (axis cs: 3.5,1) {$3$};
%%         \node at (axis cs: 3.5,3) {$4$};
%%         \node at (axis cs: 3.5,5) {$1$};
%%       \end{axis}
%%     \end{tikzpicture}
%%   \end{image}
%%   Let
%%   \[
%%   R = \{(x,y): \text{$0\le x\le 4$ and $0\le y\le 6$}\}
%%   \]
%%   \begin{prompt}
%%     \begin{align*}
%%       \Delta x &= \answer{1}\\
%%       \Delta y &= \answer{2}
%%     \end{align*}
%%   \end{prompt}
%%   Compute $\int_R F(x,y)\d A$.
%%   \begin{prompt}
%%     \[
%%     \int_R F(x,y) \d A = \answer{28}
%%     \]
%%   \end{prompt}
%% \end{question}

%% We have a theorem that helps us compute double intgerals.

%% \begin{theorem}[Fubini's Theorem]
%%   Let $F$ be continuous on the region
%%   \[
%%   R = \{(x,y):\text{$a\le x\le b$ and $c\le y\le d$}\}.
%%   \]
%%   Then:
%%   \begin{align*}
%%   \int_R F(x,y) \d A  &= \int_a^b\int_c^d F(x,y)\d y\d x\\
%%   &=\int_c^d\int_a^b F(x,y)\d x\d y.
%%   \end{align*}
%% \end{theorem}



%% Now let's work some examples:

%% \begin{example}
%%   Let $F(x,y) = xy+e^y$. Find the signed volume under $F$ on the region
%%   \[
%%   R = \{(x,y):\text{$3\le x\le4$ and $1\le y\le 2$}\}.
%%   \]
%%   \begin{image}
%%     \begin{tikzpicture}
%%       \begin{axis}%
%%         [
%%           tick label style={font=\scriptsize},%axis on top,
%% 	  axis lines=center,
%% 	  view={45}{20},
%% 	  name=myplot,
%% 	  xtick={1,2,3,4},
%% 	  %ytick={5},
%% 	  %ztick={.7,-.7},
%% 	  minor xtick=1,
%% 	  minor ytick=1,
%% 	  ymin=-.1,ymax=2.5,
%% 	  xmin=-.1,xmax=4.5,
%% 	  zmin=-.1, zmax=16,
%% 	  every axis x label/.style={at={(axis cs:\pgfkeysvalueof{/pgfplots/xmax},0,0)},xshift=-1pt,yshift=-4pt},
%% 	  xlabel={\scriptsize $x$},
%% 	  every axis y label/.style={at={(axis cs:0,\pgfkeysvalueof{/pgfplots/ymax},0)},xshift=5pt,yshift=-3pt},
%% 	  ylabel={\scriptsize $y$},
%% 	  every axis z label/.style={at={(axis cs:0,0,\pgfkeysvalueof{/pgfplots/zmax})},xshift=0pt,yshift=4pt},
%% 	  zlabel={\scriptsize $z$},
%%           colormap/cool
%% 	]        
%%         \draw [thick,penColor] (axis cs: 3,1,0) -- (axis cs: 3,2,0) -- (axis cs: 4,2,0) -- node [above,pos=.7,black] {\scriptsize $R$} (axis cs: 4,1,0) -- cycle;
        
%%         \addplot3[domain=0:4.3,,y domain=0:2.1,mesh,samples=13,samples y=18,very thin,z buffer=sort] {x*y+exp(y)};
        
%%         \addplot3[domain=1:2,%fill=white,
%%           penColor,very thick,samples=20,samples y=0] ({3},{x},{3*x+exp(x)});
        
%%         \addplot3[domain=1:2,%fill=white,
%%           penColor,very thick,samples=20,samples y=0] ({4},{x},{4*x+exp(x)});
        
%%         \addplot3[domain=3:4,%fill=white,
%%           penColor,very thick,samples=20,samples y=0] ({x},{1},{1*x+exp(1)});
        
%%         \addplot3[domain=3:4,%fill=white,
%%           penColor,very thick,samples=20,samples y=0] ({x},{2},{2*x+exp(2)});
        
%%         \draw [penColor,dashed,thin] 
%% 	(axis cs: 3,1,0) -- (axis cs: 3,1,5.7)
%% 	(axis cs: 3,2,0) -- (axis cs: 3,2,13.4)
%% 	(axis cs: 4,2,0) -- (axis cs: 4,2,15.4)
%% 	(axis cs: 4,1,0) -- (axis cs: 4,1,6.7);
%%       \end{axis}
%%     \end{tikzpicture}
%%   \end{image}
%%   \begin{explanation}
%%     We will compute this integral two different ways. Write with me,
%%     \begin{align*}
%%       \int_R \big(xy+e^y\big) \d A &= \int_3^{\answer[given]{4}}\int_{\answer[given]{1}}^{\answer[given]{2}}\big(xy+e^y\big) \d \answer[given]{y} \d \answer[given]{x}\\
%%       &= \int_{\answer[given]{3}}^{\answer[given]{4}} \eval{\answer[given]{\frac{1}{2}xy^2+e^y}}_{\answer[given]{1}}^{\answer[given]{2}} \d \answer[given]{x} \\
%%       &= \int_{\answer[given]{3}}^{\answer[given]{4}}\left(\answer[given]{\frac{3}{2}x + e^2-e}\right)\d \answer[given]{x} \\
%%       &= \eval{\answer[given]{\frac{3}{4}x^2 + (e^2-e)x}}_{\answer[given]{1}}^{\answer[given]{2}} \\
%%       &= \answer[given]{\frac{21}{4}+ e^2-e}.
%%     \end{align*}
%%   Now let's compute this integral using a different order of
%%   integration. Write with me,
%%   \begin{align*}
%%     \int_R\big(xy+e^y\big) \d A &= \int_1^{\answer[given]{2}}\int_{\answer[given]{3}}^{\answer[given]{4}}\big(xy+e^y\big)\d \answer[given]{x} \d \answer[given]{y} \\
%%     &= \int_{\answer[given]{1}}^{\answer[given]{2}}\eval{\answer[given]{\frac{1}{2}x^2y+xe^y}}_{\answer[given]{3}}^{\answer[given]{4}}\d \answer[given]{y}\\
%%     &= \int_{\answer[given]{1}}^{\answer[given]{2}}\left(\answer[given]{\frac{7}{2}y+e^y}\right)\d \answer[given]{y}\\
%%     &= \eval{\answer[given]{\frac{7}{4}y^2+e^y}}_{\answer[given]{1}}^{\answer[given]{2}}\\
%%     &=\answer[given]{\frac{21}{4}+e^2-e}.
%%   \end{align*}
%%   \end{explanation}
%% \end{example}

In our next example, we will see that it is sometimes easier to apply
Fubini's theorem and integrate with respect to one variable or the
other.

\begin{example}
  Let $F(x,y) = xe^{xy}$. Find the signed volume under $F$ on the region
  \[
  R = \{(x,y):\text{$0\le x\le 1$ and $0\le y\le 1$}\}.
  \]
  \begin{image}
    \begin{tikzpicture}
      \begin{axis}%
        [
          tick label style={font=\scriptsize},%axis on top,
	  axis lines=center,
	  view={120}{20}, %% rotate around / up and down
	  name=myplot,
	  ymin=-.1,ymax=1.1,
	  xmin=-.1,xmax=1.1,
	  zmin=-.1, zmax=3,
	  every axis x label/.style={at={(axis cs:\pgfkeysvalueof{/pgfplots/xmax},0,0)},xshift=-1pt,yshift=-4pt},
	  xlabel={\scriptsize $x$},
	  every axis y label/.style={at={(axis cs:0,\pgfkeysvalueof{/pgfplots/ymax},0)},xshift=5pt,yshift=-3pt},
	  ylabel={\scriptsize $y$},
	  every axis z label/.style={at={(axis cs:0,0,\pgfkeysvalueof{/pgfplots/zmax})},xshift=0pt,yshift=4pt},
	  zlabel={\scriptsize $z$},
          colormap/cool
	]        
        \draw [ultra thick,penColor] (axis cs: 0,0,0) -- (axis cs: 1,0,0) -- (axis cs: 1,1,0) -- (axis cs: 0,1,0) -- cycle;
        
        \node at (axis cs: .5,.5,0) {\scriptsize $R$};
        
        \addplot3[domain=0:1,,y domain=0:1,mesh,samples=13,samples y=18,very thin,z buffer=sort] {x*exp(x*y)};
        
        \addplot3[domain=0:1,%fill=white,
          penColor,very thick,samples=20,samples y=0] ({1},{x},{exp(x)});
        
        \addplot3[domain=0:1,%fill=white,
          penColor,very thick,samples=20,samples y=0] ({0},{x},{0});
        
        \addplot3[domain=0:1,%fill=white,
          penColor,very thick,samples=20,samples y=0] ({x},{0},{x});
        
        \addplot3[domain=0:1,%fill=white,
          penColor,very thick,samples=20,samples y=0] ({x},{1},{x*exp(x)});
       \end{axis}
    \end{tikzpicture}
  \end{image}
  \begin{explanation}
    Let's first compute:
    \[
    \int_0^1 \int_0^1 x e^{xy} \d x \d y
    \]
    As we will see, the diffculty is three-fold. To integrate with respect to $x$, you
    can use integration by parts to find it:
    \begin{align*}
      \int_0^1 \int_0^1 x e^{xy} \d x \d y &= \int_0^1 \eval{\answer[given]{\frac{xe^{xy}}{y}-\frac{e^{xy}}{y^2}}}_0^1\d y\\
      &= \int_0^1 \left(\frac{e^{y}}{y}-\frac{e^y}{y^2} + \frac{1}{y^2}\right)\d y
    \end{align*}
    Now we notice that this is an improper integral!
    We must therefore compute
    \[
    \lim_{b\to0}\int_{\answer[given]{b}}^{\answer[given]{1}} \left(\answer[given]{\frac{e^{y}}{y}-\frac{e^y}{y^2} + \frac{1}{y^2}}\right)\d y
    \]
    Now we must integrate with respect to $y$. This is also
    tricky. One way to do it is to rewrite our current integrand as
    \[
    \frac{y e^y - \answer[given]{(e^y -1)}}{y^2}
    \]
    and now ``see'' that this results from the quotient rule being
    applied to
    \[
    \frac{e^y-\answer[given]{1}}{y}.
    \]
    So now,
    \begin{align*}
      \lim_{b\to 0}\int_b^1 \eval{\frac{e^y-1}{y}}_b^1\\
      &= 
    \end{align*}
  \end{explanation}
\end{example}








\subsection{Triple integrals}


\[
\int_R F(x,y) \d A = \lim_{\substack{\l\to\infty\\ m\to\infty\\ n\to\infty}}\sum_{i=1}^\l\sum_{j=1}^m \sum_{k=1}^n F(x^*_i,y^*_j,z_k^*)\Delta x\Delta y\Delta z
\]


\section{Integrals with trivial integrands}

Now we will allow the region we are integrating over to be complex,
but in every case, our integrand, will be $1$.

\subsection{Double integrals}


\[
\int_R \d A = \lim_{\substack{m\to\infty\\ n\to\infty}}\sum_{i=1}^m\sum_{j=1}^n \chi_{R}(x,y)\Delta x \Delta y
\]

\subsection{Triple integrals}

\[
\int_R \d V = \lim_{\substack{\l\to\infty\\ m\to\infty\\ n\to\infty}}\sum_{i=1}^\l\sum_{j=1}^m \sum_{k=1}^n\chi_{R}(x,y,z)\Delta x\Delta y \Delta z
\]



\end{document}
