\documentclass{ximera}

\input{../preamble.tex}

\title[Dig-In:]{Introduction to multiple integrals}

\begin{document}
\begin{abstract}
NO TITLE YET.
\end{abstract}
\maketitle

BY USING THE 3D PLOT WITH A RESTRICTED DOMAIN, A FOR LOOP CAN PRODUCE THE BOXES THAT APPROX INTEGRAL


Now we move on to finding \textit{volumes} under surfaces. We start by
only considering simple regions.

\section{Integrals over trivial regions}

\begin{image}
  \begin{tikzpicture}
    \begin{axis}%
      [width=175pt,height=200pt,
        tick label style={font=\scriptsize},axis on top,
	axis lines=center,
	view={110}{25},
	name=myplot,
	%xtick=\empty,
	%ytick={5},
	%ztick={.7,-.7},
	minor xtick=1,
	minor ytick=1,
	ymin=-1.2,ymax=1.2,
	xmin=-.5,xmax=2.5,
	zmin=-.1, zmax=2.1,
	every axis x label/.style={at={(axis cs:\pgfkeysvalueof{/pgfplots/xmax},0,0)},xshift=-1pt,yshift=-4pt},
	xlabel={\scriptsize $x$},
	every axis y label/.style={at={(axis cs:0,\pgfkeysvalueof{/pgfplots/ymax},0)},xshift=5pt,yshift=-3pt},
	ylabel={\scriptsize $y$},
	every axis z label/.style={at={(axis cs:0,0,\pgfkeysvalueof{/pgfplots/zmax})},xshift=0pt,yshift=4pt},
	zlabel={\scriptsize $z$},
        colormap/cool,
      ]
      %% 12 lines
      \pgfplotsinvokeforeach{0,0.1667,...,2}{
        \draw[gray] (axis cs: #1,-1,0) -- (axis cs: #1 , 1,0);
      }

      %% 16 lines
      \pgfplotsinvokeforeach{-1,-.875,...,1}{
        \draw[gray] (axis cs: 0,#1,0) -- (axis cs: 2 ,#1,0);
      }
           

      %% Box vol
      \draw [gray,thick]
      (axis cs: 1.833,.25,0) --
      (axis cs: 1.833,.375,0) --
      (axis cs: 1.833,.375,1.665) --
      (axis cs: 1.833,.25,1.665) --
      (axis cs: 1.833,.25,0);

      \draw [gray,thick]
      (axis cs: 1.833,.375,0) --
      (axis cs: 1.833,.375,1.665) --
      (axis cs: 1.667,.375,1.665) --
      (axis cs: 1.667,.375,0) --
      (axis cs: 1.833,.375,0);

      \draw [gray,thick]
      (axis cs: 1.667,.375,1.665) --
      (axis cs: 1.667,.25,1.665) --
      (axis cs: 1.833,.25,1.665);

      %% Surface %% 12 x 16
      \addplot3[domain=0:2,y domain=-1:1,mesh,samples=13,samples y=17,very thin,z buffer=sort] {-.5*(x-1)^2-.5*(y)^2+2};

      %% Surface curves
      \addplot3[domain=0:2,%fill=white,
        penColor,very thick,samples=13,samples y=0] (
               {x},
               {1},
               {-.5*(x-1)^2-.5*(1)^2+2});

      \addplot3[domain=0:.5,%fill=white,
        penColor,very thick,dashed,samples=13,samples y=0] (
               {x},
               {-1},
               {-.5*(x-1)^2-.5*(-1)^2+2});

      \addplot3[domain=.5:2,%fill=white,
        penColor,very thick,samples=13,samples y=0] (
               {x},
               {-1},
               {-.5*(x-1)^2-.5*(-1)^2+2});

      \addplot3[domain=-1:1,%fill=white,
        penColor,very thick,samples=13,samples y=18] (
               {2},
               {y},
               {-.5*(2-1)^2-.5*(y)^2+2});

      \addplot3[domain=-1:.8,%fill=white,
        dashed,penColor,very thick,samples=13,samples y=18] (
               {0},
               {y},
               {-.5*(0-1)^2-.5*(y)^2+2});

      \addplot3[domain=.75:1,%fill=white,
        penColor,very thick,samples=13,samples y=18] (
               {0},
               {y},
               {-.5*(0-1)^2-.5*(y)^2+2});

      %% dxdydz         
      \draw [penColor, thick]
      (axis cs: 1.667,.25,1.746) --(axis cs: 1.667,.375,1.71) --
      (axis cs: 1.833,.375,1.583) --(axis cs: 1.833,.25,1.621) --
      (axis cs: 1.667,.25,1.746) -- (axis cs: 1.667,.375,1.71);
      
      
    \end{axis}
  \end{tikzpicture}
\end{image}

In the diagram above, the volume enclosed by the surface is
\[

\]


\[
\int_R F(x,y) \d A = \lim_{m\to \infty}\lim_{n\to\infty}\sum_{i=1}^m\sum_{j=1}^n F(x_i^*,y_j^*)\Delta A
\]





\section{Integrals with trivial integrands}




\end{document}
