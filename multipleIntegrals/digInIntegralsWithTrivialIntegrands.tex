\documentclass{ximera}

\input{../preamble.tex}

\outcom{Understand the definiton of a multiple integral.}

\outcome{Use iterated integrals to compute multiple integrals.}

\outcome{Apply Fubini's Theorem.}

\title[Dig-In:]{Integrals with trivial integrands}

\begin{document}
\begin{abstract}
  We study integrals over general regions, but simple integrands.
\end{abstract}
\maketitle


Now we will allow the region we are integrating over to be complex,
but in every case, our integrand, will be $1$.

\subsection{Double integrals}


\[
\int_R \d A = \lim_{\substack{m\to\infty\\ n\to\infty}}\sum_{i=1}^m\sum_{j=1}^n \chi_{R}(x,y)\Delta x \Delta y
\]

\subsection{Triple integrals}

\[
\int_R \d V = \lim_{\substack{\l\to\infty\\ m\to\infty\\ n\to\infty}}\sum_{i=1}^\l\sum_{j=1}^m \sum_{k=1}^n\chi_{R}(x,y,z)\Delta x\Delta y \Delta z
\]



\end{document}
