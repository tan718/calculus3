\documentclass{ximera}

\input{../preamble.tex}

\title[Dig-In:]{Implicit curves in space}

\begin{document}
\begin{abstract}
We learn how to plot implicit curves in space. 
\end{abstract}
\maketitle

We know that the values of $x$ and $y$ that satisfy the following
equation
\[
y=x^2
\]
will plot a parabola the $(x,y)$-plane centered at the origin. We will
use \link[\sage]{http://www.sagemath.org/} to plot this:

\begin{sageCell}
var('x,y')
implicit_plot(y==x^2, (x,-2,2),(y,-2,2),axes=true)
\end{sageCell}

Let's think deeply about what it means to ``graph'' a function. When
you graph, you
\begin{enumerate}
\item Find pairs of numbers $x$ and $y$ that satisfy a certain
  equation.
\item Take the $x$ value, multiply it by the vector $\veci$, and plot
  this point.
\item Take the $y$-value, multiply it by the vector $\vecj$, and AGGAGAGA
\end{enumerate}


So if you have an expression:

\[
F(x,y) = 0
\]
then, you compute
\[
\vector{x,y,z}\dotp 
\]

\begin{sageCell}
var('x,y')
xy=vector([x,y])
ii=vector([1/sqrt(2),1/sqrt(2)])
jj=vector([-1/sqrt(2),1/sqrt(2)])
vectori=plot(vector([1,0]),color="lightgrey")
vectorj=plot(vector([0,1]),color="lightgrey")
vectorii=plot(ii,color="dimgrey")
vectorjj=plot(jj,color="dimgrey")
curve=implicit_plot(xy.dot_product(jj)==(xy.dot_product(ii))^2, (x,-2,2),(y,-2,2))
show(vectori+vectorj+vectorii+vectorjj+curve)
\end{sageCell}


\begin{sageCell}
var('x,y,z')
xy=vector([x,y])
ii=vector([1,0,0])
jj=vector([0,1,0])
kk=vector([0,0,1])
surface1=implicit_plot3d(x^2+y^2+z^2==3, (x,-2,2),(y,-2,2),(z,-2,2))
surface2=implicit_plot3d(z==0, (x,-2,2),(y,-2,2),(z,-2,2),color="red")
show(surface1+surface2)
\end{sageCell}




\end{document}
