\documentclass{ximera}

\input{../preamble.tex}

\title[Dig-In:]{Planes in space}

\begin{document}
\begin{abstract}
  We discuss how to find implicit and explicit formulas for planes.
\end{abstract}
\maketitle

Planes are the three-dimensional analogue of lines in two-dimensions.

\section{Implicit planes}

Remember an implicit function in $\R^3$ is one of the form:
\[
F(x,y,z) = 0
\]
We would like to know the implicit formula for a plane. Here the dot
product saves the day. Recall that if $\vec{v} = \vector{a,b,c}$ is
any vector, and $\vec{x}= \vector{x,y,z}$, then the equation
\[
\vec{v}\dotp\vec{x} = 0
\]
is solved by all vectors $\vec{x}$ that are orthogonal to
$\vec{v}$. We plotted several such vectors below:
\begin{image}
          \begin{tikzpicture}
          \begin{axis}%
            [tick label style={font=\scriptsize},axis on top,
	      axis lines=center,
	      view={135}{25},
	      name=myplot,
	      %xtick={-3,3},minor tick num=2,
	      %ytick={-3,3},
	      %ztick={-3,3},
	      ymin=-4,ymax=4,
	      xmin=-4,xmax=4,
	      zmin=-4, zmax=4,
	      every axis x label/.style={at={(axis cs:\pgfkeysvalueof{/pgfplots/xmax},0,0)},xshift=-3pt,yshift=-3pt},
	      xlabel={\scriptsize $x$},
	      every axis y label/.style={at={(axis cs:0,\pgfkeysvalueof{/pgfplots/ymax},0)},xshift=0pt,yshift=-5pt},
	      ylabel={\scriptsize $y$},
	      every axis z label/.style={at={(axis cs:0,0,\pgfkeysvalueof{/pgfplots/zmax})},xshift=0pt,yshift=4pt},
	      zlabel={\scriptsize $z$}
	    ]
            \draw[thick,->,penColor2] (axis cs: 2,1,1) -- (axis cs: 2/3,2/3,2/3);
            \draw[thick,->,penColor2] (axis cs: 2,1,1) -- (axis cs: 1.057,.057,1.47);
            \draw[thick,->,penColor2] (axis cs: 2,1,1) -- (axis cs: 2,0,2);
            \draw[thick,->,penColor2] (axis cs: 2,1,1) -- (axis cs: 2.943,.529,1.943);
            \draw[thick,->,penColor2] (axis cs: 2,1,1) -- (axis cs: 10/3,4/3,4/3);
            \draw[thick,->,penColor2] (axis cs: 2,1,1) -- (axis cs: 2.943,1.943,0.529);
            \draw[thick,->,penColor2] (axis cs: 2,1,1) -- (axis cs: 2,2,0);
            \draw[thick,->,penColor2] (axis cs: 2,1,1) -- (axis cs: 1.057,1.471,0.057);
            
            
            \draw[thick,->,penColor] (axis cs: 2,1,1) -- (axis cs: 1,3,3);
          \end{axis}
        \end{tikzpicture}
\end{image}
From this we see that
\begin{align*}
\vec{v}\dotp\vec{x} &=0\\
ax+by+cz &= 0
\end{align*}
gives the formula for a plane. Since $\vec{0} = \vec{x}$ is a
solution, this plane must pass through the origin. If we want our
plane to be located anywhere in space, we must know a point on the
plane, call it $\vec{p}=\vector{x_0,y_0,z_0}$. Putting this together, we can
now see that if you know
\begin{itemize}
  \item a vector $\vec{v} = \vector{a,b,c}$ and
  \item a point (given by a vector) $\vec{p} = \vector{x_0,y_0,z_0}$
\end{itemize}
then,
\begin{align*}
  (\vec{v}-\vec{p})\dotp \vec{x} &= 0\\
  a(x-x_0) + b(y-y_0) + c(z-z_0) &= 0
\end{align*}
is a formula for a plane passing through the point $(x_0,y_0,z_0)$
with normal vector $\vec{v}$.

\begin{question}
  Find the implicit equation of a plane that passes through the point
  $(5,-5,-1)$ and with normal vector $\vector{-5,5,5}$.
  \begin{onlineOnly}
    Check your answer by modifiying the definition of the \sage code
    for $F(x,y,z)$ below:
  \begin{sageCell}
vector=arrow3d((3,-5,-1),(-2,0,4),5,color='blue');

F(x,y,z) = -5*(x-3)+5*(y+5)+5*(z+1)

plane=implicit_plot3d(F(x,y,z)==1,(x,-10,10),(y,-10,10),(z,-10,10),color='red');
plane+vector
  \end{sageCell}
  \end{onlineOnly}
\end{question}



\section{Parametric planes}

\end{document}
