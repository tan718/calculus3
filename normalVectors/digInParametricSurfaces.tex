\documentclass{ximera}

\input{../preamble.tex}

\title[Dig-In:]{Parametric surfaces}

\begin{document}
\begin{abstract}
  Tangent and normal vectors can help model surfaces in space.
\end{abstract}
\maketitle

Unit tangent vectors and unit normal vectors can help us use
mathematics to model surfaces in space.

\section{Thickened curves}

\begin{example}[Thickening a curve]
  You have been given a curve in space, meaning that you have been given a function
  \begin{multipleChoice}
    \choice{$f: \R \to \R$}
    \choice{$F: \R^3 \to \R$}
    \choice[correct]{$\vec{f}: \R \to \R^3$}
  \end{multipleChoice}
  and your task is to build a parameterized surface which is a ``thickened'' version.

  In other words, we want to convert a curve like
  \begin{sageOutput}
t = var('t')
x = sin(t)
y = sin(2*t + pi/5)
z = sin(3*t + pi/7)
f = vector([x,y,z])
parametric_plot3d( f, (t,0,2*pi) )
  \end{sageOutput}
  into a thickened ``tube'' like
  \begin{sageOutput}
t = var('t')
x = sin(t)
y = sin(2*t + pi/5)
z = sin(3*t + pi/7)
f = vector([x,y,z])
df = derivative(f,t)
ut = df / df.norm()
ddf = derivative(ut,t)
n = ddf / ddf.norm()
bn = (n.cross_product(ut)) / ( (n.cross_product(ut)).norm() )
thickness = 0.10
s = var('s')
parametric_plot3d( f + (n * cos(s) + bn * sin(s)) * thickness, (t,0,2*pi), (s,0,2*pi), plot_points=[100,100] )
  \end{sageOutput}

  Our tube around $\vec{f}$ will be a parametrized surface given by 
  \begin{multipleChoice}
    \choice{$\vec{f}: \R \to \R^2$}
    \choice{$F: \R^2 \to \R$}
    \choice[correct]{$\vec{F}: \R^2 \to \R^3$}
    \choice{$\vec{F}: \R^3 \to \R^2$}
  \end{multipleChoice}

  We can parametrize a circle with
  \begin{multipleChoice}
    \choice{$t \mapsto (x^2, y^2)$}
    \choice{$t \mapsto (\cos^2(t), \sin^2(t))$}
    \choice[correct]{$t \mapsto (\cos(t), \sin(t))$}
  \end{multipleChoice}
  so we could try
  \[
    \vec{F}(t,s) = \vec{f}(t) + 0.25 \langle \cos s, \sin s, 0 \rangle.
  \]
  Note that we are using \textbf{vector operations}---we are taking the
\wordChoice{\choice[correct]{sum}\choice{dot product}} of two vectors to build our parametrized surface.  We want the vector $\vec{f}(t)$ to sit \wordChoice{\choice{on the surface}\choice[correct]{inside the tube}} and the second term pushes us away from $\vec{f}(t)$ in a different direction as $s$ varies.  The factor of 0.25 is included to \wordChoice{\choice[correct]{shrink the circle}\choice{expand the circle}}.  Putting this together, we can ``Evaluate'' to see a picture:
  \begin{sageCell}
t = var('t')
x = sin(t) ; y = sin(2*t + pi/5) ; z = sin(3*t + pi/7)
f = vector([x,y,z])

s = var('s')
parametric_plot3d( f + 0.25 * vector([cos(s), sin(s), 0]), (t,0,2*pi), (s,0,2*pi) )
\end{sageCell}
and that doesn't look awful, but it also certainly doesn't quite look
right!  The trouble is the circle we sweep out with
$\langle \cos s, \sin s, 0 \rangle$ is not always
\wordChoice{\choice[correct]{perpendicular}\choice{parallel}} to the
unit tangent vector
\begin{multipleChoice}
  \choice{$\utan(t) = \frac{\vec{f}'(t)}{|\vec{f}'(t)|^2}$}
  \choice[correct]{$\utan(t) = \frac{\vec{f}'(t)}{|\vec{f}'(t)|}$}
\end{multipleChoice}

To improve our tube, we want to sweep out a circle in the plane given by
\begin{multipleChoice}
  \choice{$\utan(t)$ and $\unormal(t)$}
  \choice{$\utan(t)$ and $\unormal(t) \times \utan(t)$}
  \choice[correct]{$\unormal(t)$ and $\unormal(t) \times \utan(t)$}
\end{multipleChoice}
Let $\vec{b} = \unormal(t) \times \utan(t)$ and $\ubinormal = \frac{\vec{b}}{| \vec{b} |}$.

Then we can sweep out a tube with
  \[
    \vec{F}(t,s) = \vec{f}(t) + 0.25 \left( \left( \cos s \right) \unormal(t) + \left( \sin s \right) \ubinormal(t) \right).
  \]
  Try it!
\begin{sageCell}
t = var('t')
x = sin(t) ; y = sin(2*t + pi/5) ; z = sin(3*t + pi/7)

f = vector([x,y,z])
df = derivative(f,t)
u = df.normalized()
n = derivative(u,t).normalized()
b = n.cross_product(u).normalized()

thickness = 0.25
s = var('s')
parametric_plot3d( f + (n * cos(s) + b * sin(s)) * thickness, (t,0,2*pi), (s,0,2*pi), plot_points=[100,100] )
\end{sageCell}

\end{example}  

\section{Mirrored surfaces}

SPOON

PARABOLIC
upside down




\end{document}
