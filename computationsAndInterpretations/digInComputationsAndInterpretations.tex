\documentclass{ximera}

\input{../preamble.tex}

\title[Dig-In:]{Computations and interpretations}

\begin{document}
\begin{abstract}
  We practice more computations and think about what integrals mean.
\end{abstract}
\maketitle

In this section we will continue to set-up (and sometimes compute)
double and triple integrals and think about what these mean.

\section{Triangles}

\begin{example}
  Let $T$ be the triangle with vertices $(0,0)$ and $(\pi,0)$ and
  $(\pi,\pi)$.  Then we can write $T$ as
  \[
  T = \{ (x,y)  : 0 \leq y \leq \pi \text{ and } \answer[given]{y} \leq x \leq \answer[given]{\pi} \},
  \]
  but we could also write
  \[
    T = \{ (x,y)  : 0 \leq x \leq \pi \text{ and } \answer[given]{0} \leq y \leq \answer[given]{x} \}.
  \]
  Which of these might be a better choice to show that $\displaystyle\int_T 2 \sin (x^2) \d A = \answer[given]{1 - \cos(\pi^2)}$?
  
  \begin{explanation}
    Based on these two different descriptions of $T$, we can evaluate
    the double integral $\int_T 2 \sin (x^2) \d A$ through two rather
    different looking iterated integrals.  To wit,
    \begin{align*}
      \int_T 2 \sin (x^2) \d A 
      &= \int_{y = \answer{0}}^{\answer{\pi}} \int_{x = y}^\pi 2 \sin (x^2) \d x \d y \\
      &= \int_{x = 0}^\pi \int_{y = \answer{0}}^{\answer{x}} 2 \sin (x^2) \d y \d x.
    \end{align*}
    If we were to use the first description of $T$, we might have
    trouble finding an $x$-antiderivative of $\sin (x^2)$, so let's
    try the second description.  In that case,
    \[
      \int_T 2 \sin (x^2) \d A = \int_{x = 0}^\pi 2 x \sin (x^2) \d x,
    \]
    nd now let $u = x^2$ so $du = \answer{2x} \d x$, so 
    \begin{align*}
      \int_T 2 \sin (x^2) \d A 
      &= \int_{u = 0}^{\pi^2} \sin u \d u \\
      &= \left( \answer{- \cos u} \right)_{u=0}^{\pi^2} \\
      &= \answer{1 - \cos (\pi^2)}.
    \end{align*}
  \end{explanation}
\end{example}

\begin{question}
  It's important to do a self-check to see if our purported value for
  an integral is at all plausible.

  The region $T$ is a triangle with base $\answer{\pi}$ and height
  $\answer{\pi}$, so the area of the region $T$ is $\answer{\pi^2/2}$
  which is about $5$ square units.  In other words,
  \[
    \int_T 1 \d A = \answer{\pi^2/2} \approx 5,
  \]
  which also means that
  \[
    \int_T 2 \d A = \answer{\pi^2} \approx 10.
  \]
  We are claiming that $\int_T 2 \sin (x^2) \d A$ equals
  $1 - \cos (\pi^2)$, which is about $1.9$.

  When $0 \leq x \leq \pi$, the value of $\sin (x^2)$ is sometimes
  positive, sometimes negative, but at least we know that
  \[
    -2 \leq 2 \sin (x^2) \leq 2,
  \]
  and this inequality then implies that
  \[
    \left| \int_T 2 \sin (x^2) \d A \right| \leq \left| \int_T 2 \d A \right| \answer{\pi^2}.
  \]
  So $1.9$ is certainly in the ballpark of plausibility.
\end{question}

\section{Polar coordinates}

\begin{example}
  Evaluate the integral $\int_Q \left(x + y\right) \d A$ where $Q$ is the quarter circle,
  \[
    Q = \{ (x,y)  : x \geq 0,\hspace{1ex} y \geq 0,\hspace{1ex} x^2 + y^2 \leq 1 \}.
  \]

  \begin{explanation}
    We use \wordChoice{\choice{rectangular}\choice[correct]{polar}} coordinates.
    \begin{align*}
      \int_Q \left(x + y\right) \d A
      &= \int_{\theta=\answer{0}}^{\answer{\pi/2}} \int_{r=0}^{1} (r \cos \theta + r \sin \theta) \, r \d r \d \theta \\  
      &= \left( \int_{r=0}^{1} r^2 \d r \right) \cdot \left( \int_{\theta=0}^{\pi/2} (\cos \theta + \sin \theta) \d \theta \right) \\  
      &= \frac{1}{3}  \cdot \left(\sin \theta - \cos \theta\right)_{\theta=0}^{\pi/2} \\
      &= \frac{1}{3} \cdot 2 = \answer{2/3}.
    \end{align*}
  \end{explanation}
\end{example}

\begin{question}
  Again consider the region
  \[
    Q = \{ (x,y)  : x \geq 0,\hspace{1ex} y \geq 0,\hspace{1ex} x^2 + y^2 \leq 1 \}.
  \]
  How does
  \[
    A = \int_Q x \d A  
  \]
  compare to 
  \[
    B = \int_Q y \d A?
  \]

  \begin{multipleChoice}
    \choice{$A < B$}
    \choice[correct]{$A = B$}
    \choice{$A > B$}
  \end{multipleChoice}

  \begin{feedback}[correct]
    The region $Q$ is symmetric across the line $x = y$.  As a consequence of this, we might have computed that
    \[
      A = \int_Q x \d A = 1/3,
    \]
    and likewise $B = 1/3$.  Because of this---and the fact that the integral of a sum is the sum of integrals---we could have deduced
    \begin{align*}
      A + B &= (1/3) + (1/3) \\
            &= \int_Q x \d A + \int_Q y \d A \\
            &= \int_Q (x+y) \d A = 2/3.
    \end{align*}
  \end{feedback}
\end{question}

\section{Spheres and hemispheres}

\begin{example}
  Let $B$ be the region $B = \{ (x,y,z)  : x^2 + y^2 + z^2 \leq 1\}$.

  Explain why $\displaystyle\iiint_B z^3 \d V = \answer[given]{0}$.

  \begin{explanation}
    This integral vanishes because integral over the northern
    hemisphere of $B$ will cancel the contribution from the southern
    hemisphere of $B$.

    Consider the function $f : \R^3 \to \R^3$ given by
    $f(x,y,z) = (x,y,-z)$.  This function exchanges the northern and
    southern hemispheres, but the effect of $f$ on the integrand is to
    negate it!
  \end{explanation}
\end{example}

\begin{example}
  Let $H$ be the region $H = \{ (x,y,z)  : x^2 + y^2 + z^2 \leq 1 \text{ and } z \geq 0 \}$.

  Show that $\displaystyle\int_H z^3 \d V = \answer[given]{\pi/12}$.

  \begin{explanation}
    Unlike the previous example, this does not vanish.

    We use \wordChoice{\choice{cylindrical}\choice[correct]{spherical}} coordinates.
    \begin{align*}
      \int_D z^3 \d V
      &= \int_{\theta=0}^{2\pi} \int_{\varphi=0}^{\pi/2} \int_{r = 0}^1 (r \cos \varphi)^3 \, r^2 \sin \varphi \d r \d \varphi \d \theta \\
      &= \int_{\theta=0}^{2\pi} \int_{\varphi=0}^{\pi/2} \int_{r = 0}^1 r^5 \cos^3 \varphi \sin \varphi \d r \d \varphi \d \theta \\
      &= \left( \int_{\theta=0}^{2\pi} d\theta \right) \left(\int_{\varphi=0}^{\pi/2} \cos^3 \varphi \sin \varphi \d \varphi \right) \left( \int_{r = 0}^1 r^5 \d r \right) \\
      &= 2 \pi \left(\int_{\varphi=0}^{1} u^3 \d u \right) \frac{1}{6} \\
      &= 2 \pi \frac{1}{4} \cdot \frac{1}{6} \\
      &= frac{\pi}{12}.
    \end{align*}
  \end{explanation}
\end{example}

\begin{question}
  Again let $H$ be the region $H = \{ (x,y,z): \text{$x^2 + y^2 +
  z^2 \leq 1$ and $z \geq 0 $}\}$.

  Set $A = \int_H z^3 \d V$ and $B = \displaystyle\int_H z^{10} \d V$.  How does $A$ relate to $B$?

  \begin{multipleChoice}
    \choice{$A < B$}
    \choice{$A = B$}
    \choice[correct]{$A > B$}
  \end{multipleChoice}

  \begin{feedback}[correct]
    Indeed, $A > B$ because, for points $(x,y,z) \in H$, we have
    $-1 \leq z \leq 1$ and $z^3 \geq z^{10}$.  Moreover, except when
    $z = \pm 1$, it is the case that $z^3 > z^{10}$.  By comparing the
    integrands, we can gain insight into the relative sizes of the
    integrals.

    This same kind of thinking can lend insight into the question of
    what happens when $\displaystyle\int_H z^N \d V$ when $N$ is very
    large.
  \end{feedback}
\end{question}

\end{document}
